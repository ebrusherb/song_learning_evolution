\documentclass{article}

\usepackage[textwidth=7in,textheight=10in]{geometry}
\usepackage{graphicx}
\usepackage{enumerate}
\usepackage{/Users/eleanorbrush/Documents/custom2}
\usepackage{wasysym}
\usepackage{color}
\usepackage[numbers,sort&compress]{natbib}
\newcommand{\ra}[1]{\renewcommand{\arraystretch}{#1}}
\newcommand{\x}[1]{\text{#1}}
\newcommand{\Cov}{\text{Cov}}
\usepackage{lscape} 
\newcommand\numberthis{\addtocounter{equation}{1}\tag{\theequation}}

\usepackage{tocloft}% http://ctan.org/pkg/tocloft
\setlength{\cftsecnumwidth}{2em}% Set length of number width in ToC for \subsection
\cftsetindents{subsection}{3em}{2em}
\setcounter{tocdepth}{2}% Allow only \chapter in ToC

\begin{document}
%\tableofcontents

\section*{Introduction}
We find that 
\begin{enumerate}
\item There are more cases in which variance in the distribution of songs can be maintained when song is genetic as opposed to paternally learned.
\item The actual level of variance in the distribution of songs is higher in those cases when song is paternally learned as opposed to genetic.
\item There is the most variance in the distribution of songs when females are highly choosy when deciding with whom to mate.
\item The basic rule dictating how mating pairs form allows for both a normal distribution of songs at equilibrium and a multimodal distribution of songs at equilibrium.
\end{enumerate}

\section*{Model}

In our model, we consider a single population of birds. Each male bird has a song it uses to attract female birds. Each female has a preference for a particular song. All females get to mate and have equal mating success. On the other hand, males can mate with multiple females and those with more attractive songs will have higher mating success. Depending on the scenarios we consider, both male song and female preference may be genetically inherited or learned. If the song is genetic, then both males and females carry a gene for song, and females may in fact sing, but a bird's song only affects the mating success of males. Male birds may learn their songs from their father, or they may learn obliquely, learning a randomly chosen song from all songs present in the population. If the preference is genetic, then both males and females carry a gene for preference, but a male bird's preference gene has no effect on who it mates with. After the adults mate and reproduce, they die and their offspring become next year's adults. In other words, we assume there are non-overlapping generations. Depending on how songs and preferences are acquired, the population distribution of these traits will change over time. In particular, the diversity of songs among offspring can either increase or decrease over time.

\subsection*{Mating. }
\ Mathematically, we will assume every bird has two traits, a song $x$ and preference $y$. (Table \ref{variables} defines all the variables used in the text.) We assume that initially both traits are normally distributed among the adults of the population. 
If $v=(x,y)^T$ is the two-dimensional vector giving the song and preference traits, the probability distribution of these traits among all adult males is 
\begin{align*}
P_\x{m}(v)&=\frac{1}{2\pi\sqrt{|\Sigma_\x{m}|}}\exp\left(-\frac{1}{2}(v-\mu_\x{m})^T\Sigma_\x{m}^{-1}(v-\mu_\x{m})\right),
\end{align*} where $\mu_\x{m}=(\mu_{x\x{m}},\mu_{y\x{m}})^T$ gives the expected values of songs and preferences among adult males and 
\begin{align*}
\Sigma_{\x{m}}=\left(\begin{array}{cc}\sigma_{x\x{m}}^2 & \rho_\x{m}\sigma_{x\x{m}}\sigma_{y\x{m}} \\ \rho_\x{m}\sigma_{x\x{m}}\sigma_{y\x{m}} & \sigma_{y\x{m}}^2 \end{array}\right)
\end{align*}
is the covariance matrix for the traits in adult males. Note that the parameter $\rho_\x{m}$ is the correlation between songs and preferences among adult males. We will also use $C_\x{m}=\rho_\x{m}\sigma_{x\x{m}}\sigma_{y\x{m}}$ to denote the covariance between the traits in adult males. Similarly, the probability distribution of these traits among all adult females is 
\begin{align*}
P_\x{f}(v)&=\frac{1}{2\pi\sqrt{|\Sigma_\x{f}|}}\exp\left(-\frac{1}{2}(v-\mu_\x{f})^T\Sigma_\x{f}^{-1}(v-\mu_\x{f})\right), 
\end{align*}
where $\mu_\x{f}$ gives the expected values of songs and preferences among adult females,  $\Sigma_\x{f}$ is the covariance matrix of these traits among adult females, $\rho_\x{f}$ is the correlation between the traits among adult females, and $C_\x{f}$ is the covariance between the trait in adult females. For most of the analyses that follow, we assume that each female uses a Gaussian preference function, centered at her preference $y$ with a variance $\sigma^2$:
\begin{align*}
f_y(x)&=\frac{1}{\sqrt{2\pi\sigma^2}}\exp\left(\frac{(x-y)^2}{2\sigma^2}\right).
\end{align*}
The variance $\sigma^2$ can be thought of as promiscuity---the larger $\sigma^2$ is, the more males a female is willing to mate with--- or as being inversely related to choosiness---the larger $\sigma^2$ is, the less choosy a female is. 

The probability of a male with traits $v_\x{m}$ and a female with traits $v_\x{f}$ mating is proportional to the product of the probabilities of finding such a male and such a female, with an additional factor describing the likelihood of such a female mating with such a male:
\begin{equation} \label{model}
P_\x{mate}(v_\x{m},v_\x{f})=\frac{P_\x{f}(v_\x{f})P_\x{m}(v_\x{m})f_{y_\x{f}}(x_\x{m})}{Z_{y_\x{f}}},
\end{equation}
where $Z_{y_\x{f}}=\int_{\R^2} P_\x{m}(v_\x{m})f_{y_\x{f}}(x_\x{m})dv_\x{m}$ is a normalizing factor that describes the fraction of males a female with preference $y_\x{f}$ finds attractive.

In the Supporting Information, we derive the probability that a male with traits $v_\x{m}=(x_\x{m},y_\x{m})^T$ and a female with traits $v_\x{f}=(x_\x{f},y_\x{f})^T$ will mate. Because we use normal distributions for the initial conditions of both traits and for the preference function, the probability of a particular pair mating can also be described with a normal distribution. Specifically, if $u=(x_\x{m},y_\x{m},x_\x{f},y_\x{f})^T$ the distribution $P_\text{mate}(u)$ is a multivariate normal with expectation 
\begin{align}
\mu_\text{mate}&=\left(\begin{array}{cc} \frac{\sigma^2}{\sigma^2+\sigma_{x\x{m}}^2}\mu_{x\x{m}}+\frac{\sigma_{x\x{m}}^2}{\sigma^2+\sigma_{x\x{m}}^2}\mu_{y\x{f}} 
\\ \mu_{y\x{m}}+\frac{\rho_\x{m}\sigma_{x\x{m}}\sigma_{y\x{m}}}{\sigma^2+\sigma_{x\x{m}}^2}(\mu_{y\x{f}}-\mu_{x\x{m}})
\\ \mu_{x\x{f}}
\\ \mu_{y\x{f}}
 \end{array}\right) \label{mean}
\end{align}
and covariance $\Sigma_\text{mate}$:
\begin{align}
\left(\begin{array}{cccc}\sigma_{x\x{m}}^2\left(\frac{\sigma^2(\sigma^2+\sigma_{x\x{m}}^2)+\sigma_{x\x{m}}^2\sigma_{y\x{f}}^2}{(\sigma^2+\sigma_{x\x{m}}^2)^2}\right) & \frac{\rho_\x{m}\sigma_{x\x{m}}\sigma_{y\x{m}}(\sigma^2(\sigma^2+\sigma_{x\x{m}}^2)+\sigma_{x\x{m}}^2\sigma_{y\x{f}}^2)}{(\sigma^2+\sigma_{x\x{m}}^2)^2} & \sigma_{x\x{m}}^2\frac{\rho_\x{f}\sigma_{x\x{f}}\sigma_{y\x{f}}}{\sigma^2+\sigma_{x\x{m}}^2} &  \sigma_{x\x{m}}^2\frac{\sigma_{y\x{f}}^2}{\sigma^2+\sigma_{x\x{m}}^2}
\\ & \sigma_{y\x{m}}^2\left(\frac{(\sigma^2+\sigma_{x\x{m}}^2)^2-\rho_\x{m}^2\sigma_{x\x{m}}^2(\sigma^2+\sigma_{x\x{m}}^2)+\rho_\x{m}^2\sigma_{x\x{m}}^2\sigma_{y\x{f}}^2}{(\sigma^2+\sigma_{x\x{m}}^2)^2}\right) & \frac{\rho_\x{m}\sigma_{x\x{m}}\sigma_{y\x{m}}\sigma_{x\x{f}}\sigma_{y\x{f}}}{\sigma^2+\sigma_{x\x{m}}^2}& \sigma_{y\x{f}}^2\frac{\rho_\x{m}\sigma_{x\x{m}}\sigma_{y\x{m}}}{\sigma^2+\sigma_{x\x{m}}^2}
\\ & & \sigma_{x\x{f}}^2 & \rho_\x{f}\sigma_{x\x{f}}\sigma_{y\x{f}}
\\ & & & \sigma_{y\x{f}}^2
\end{array}\right). \label{covariance}
\end{align}
Note that both the expected values and the covariance structure of the traits among  mating females is the same as among all adult females. This is because of our assumption that all females have equal reproductive success. Among mating males, however, the expected song becomes a weighted average of the average male song and average female preference. The expected male preference increases by an amount that depends on the correlation between male song and male preference and on whether the average female preference is greater than or less than the average male song.

\subsection*{Transmission of traits to offspring.  } 
\ This distribution of the traits within mating pairs allows us to find the distribution of the traits within the offspring of these pairs. To do so, we need to specify how the offspring acquire each trait. We will consider three ways each trait can be acquired. The song can be acquired by 
\begin{enumerate}[A.]
\item a male learning the song of a randomly chosen adult (``obliquely" as opposed to vertically),
\item a bird of either sex inheriting the song gene from its parents, 
\item or a male learning the song of its father.
\end{enumerate}
When song is genetic, we will assume that a bird's song is the average of its parents' songs. Since the distribution of mating pairs is normal, the distribution of the traits among offspring will also be normal. For example, using Equations \ref{mean} and  \ref{covariance}, if a male learns the song of its father, the songs among males in the offspring generation will be normally distributed with mean $$\mu_{x\x{m}}(t+1)=\frac{\sigma^2}{\sigma^2+\sigma_{x\x{m}}^2}\mu_{x\x{m}}+\frac{\sigma_{x\x{m}}^2}{\sigma^2+\sigma_{x\x{m}}^2}\mu_{y\x{f}}$$ and variance $$\sigma_{x\x{m}}^2(t+1)=\sigma_{x\x{m}}^2\left(\frac{\sigma^2(\sigma^2+\sigma_{x\x{m}}^2)+\sigma_{ x\x{m}}^2\sigma_{y\x{f}}^2}{(\sigma^2+\sigma_{x\x{m}}^2)^2}\right).$$
We focus on how the variances of the two traits change over time. For the three ways of acquiring song, the variance of songs among males of the offspring generation obey the following recursion equations: 
\begin{align*}
\x{A.} \ &\sigma_{x\x{m}}^2(t+1)=\sigma_{x\x{m}}^2
\\ \x{B.} \ &\sigma_{x\x{m}}^2(t+1)=\frac{1}{4}\sigma_{x\x{m}}^2\left(\frac{\sigma^2(\sigma^2+\sigma_{x\x{m}}^2)+\sigma_{ x\x{m}}^2\sigma_{y\x{f}}^2}{(\sigma^2+\sigma_{x\x{m}}^2)^2}\right)+\frac{1}{2}\sigma_{x\x{m}}^2\frac{\rho_\x{f}\sigma_{x\x{f}}\sigma_{y\x{f}}}{\sigma^2+\sigma_{x\x{m}}^2}+\frac{1}{4}\sigma_{x\x{f}}^2
\\\x{C.} \ &\sigma_{x\x{m}}^2(t+1)=\sigma_{x\x{m}}^2\left(\frac{\sigma^2(\sigma^2+\sigma_{x\x{m}}^2)+\sigma_{ x\x{m}}^2\sigma_{y\x{f}}^2}{(\sigma^2+\sigma_{x\x{m}}^2)^2}\right).
\end{align*}
When males acquire their song by learning from a randomly chosen adult male, the distribution of songs does not change over time, so the variance of songs stays the same. 

The preference can be acquired by 
\begin{enumerate}[I.]
\item a female learning the preference of its mother,
\item a bird of either sex inheriting the preference gene from its parents,
\item or a female learning the song of its father and preferring that song.
\end{enumerate}
As with song, when preference is genetic, we will assume that a bird's preference is the average of its parents' preferences. Using Equation 2, we can also find the variance of preferences among females of the offspring generation for the three ways of acquiring preference:
\begin{align*}
\x{I.} \ & \sigma_{y\x{f}}^2(t+1) = \sigma_{y\x{f}}^2
\\\x{II.} \ & \sigma_{y\x{f}}^2(t+1) = \frac{1}{4}\sigma_{y\x{m}}^2\left(\frac{(\sigma^2+\sigma_{x\x{m}}^2)^2-\rho_\x{m}^2\sigma_{x\x{m}}^2(\sigma^2+\sigma_{x\x{m}}^2)+\rho_\x{m}^2\sigma_{x\x{m}}^2\sigma_{y\x{f}}^2}{(\sigma^2+\sigma_{x\x{m}}^2)^2}\right)+\frac{1}{2}\sigma_{y\x{f}}^2\frac{\rho_\x{m}\sigma_{x\x{m}}\sigma_{y\x{m}}}{\sigma^2+\sigma_{x\x{m}}^2}+\frac{1}{4}\sigma_{y\x{f}}^2
\\\x{III.} \ & \sigma_{y\x{f}}^2(t+1) = \sigma_{x\x{m}}^2\left(\frac{\sigma^2(\sigma^2+\sigma_{x\x{m}}^2)+\sigma_{ x\x{m}}^2\sigma_{y\x{f}}^2}{(\sigma^2+\sigma_{x\x{m}}^2)^2}\right).
\end{align*}
When females acquire their song by learning from their mothers, the distribution of songs does not change over time because all females have equal mating success, so the variance of preferences stays the same. Note that this could happen if females observe the mating choices of their mothers and copy those choices or if preference is a sex-linked gene that only females have. 

Given a distribution of traits among adults, how the variance of songs changes does not depend on how preference is acquired. Likewise, how the variance of preferences changes does not depend on how song is acquired. However, in order to keep track of how these quantities change over many generations, we also need to keep track of how the correlations $\rho_\x{m}$ and $\rho_\x{f}$, or equivalently the covariances $C_\x{m}$ and $C_\x{f}$, change over time. These dynamics depend both on how songs are acquired and on how preferences are acquired. Note that there can only be a correlation between the traits if at least one is genetic because otherwise each sex only has one of the two traits.

There are nine possible combinations of the three ways each trait can be acquired, which we will number from $1$ to $9$: 
\newline
\begin{equation*}
\begin{tabular}{|l|c|c|c|}
\hline & A. song learned obliquely  & B. song genetic & C. song learned from father
\\\hline I. pref learned from mother & 1 & 2 & 3
\\\hline II. pref genetic & 4 & 5 & 6
\\\hline III. pref learned from father & 7 & 8 & 9
\\\hline
\end{tabular}
\end{equation*}
\newline

In the Supporting Information, we derive the full dynamics of $\sigma_x^2$, $\sigma_y^2$, and $C$ for each of these modes of acquisition. Here we will show as an example mode $2$, the case where song is genetic and a female acquires the preference of its mother. Because both sexes acquire song in the same way, we can drop the sex-specific subscripts on $\sigma_x^2$, and because only females have preferences, we can drop the sex-specific subscripts on $\sigma_y^2$ without losing track of which sex we are referring to. In this case, because males only have songs, $C_\x{m}=0$. However, females have both traits and, using Equation \ref{covariance}, the covariance between those traits among females of the next generation will be 
\begin{align*}
C_\x{f}(t+1)&=\frac{1}{2}\sigma_{x}^2\frac{\sigma_{y}^2}{\sigma^2+\sigma_{x}^2}+\frac{1}{2}\rho_\x{f}\sigma_{x}\sigma_{y}.
\end{align*}
Since 
Using the fact that $C_\x{f}=\rho_\x{f}\sigma_{x}\sigma_{y}$, we now have three recursion equations for how $\sigma_{x}^2$, $\sigma_{y}^2$, and $C_\x{f}$ change over time:
\begin{align*}
\sigma_x^2(t+1)&=\frac{\sigma_x^2}{4}\left(\frac{\sigma^2(\sigma^2+\sigma_{x}^2)+\sigma_{x}^2\sigma_{y}^2}{(\sigma^2+\sigma_{x}^2)^2}+\frac{2C_\x{f}}{\sigma^2+\sigma_x^2}+1\right)
\\ \sigma_y^2(t+1)&=\sigma_y^2
\\ C_\x{f}(t+1)&=\frac{1}{2}\sigma_{x}^2\frac{\sigma_{y}^2}{\sigma^2+\sigma_{x}^2}+\frac{1}{2}C_\x{f}.
\end{align*}
%Consider the case where song is genetic and preference is learned from a female's father. In this case, males only have songs, so $C_\x{m}=0$. However, females have both traits and, using Equation \ref{covariance}, the covariance between those traits among females of the next generation will be 
%\begin{align*}
%C_\x{f}(t+1)&=\frac{1}{2}\sigma_{x\x{m}}^2\left(\frac{\sigma^2(\sigma^2+\sigma_{x\x{m}}^2)+\sigma_{x\x{m}}^2\sigma_{y\x{f}}^2}{(\sigma^2+\sigma_{x\x{m}}^2)^2}\right)+\frac{1}{2}\sigma_{x\x{m}}^2\frac{\rho_\x{f}\sigma_{x\x{f}}\sigma_{y\x{f}}}{\sigma^2+\sigma_{x\x{m}}^2}.
%\end{align*}
%Using the fact that $C_\x{f}=\rho_\x{f}\sigma_{x\x{f}}\sigma_{y\x{f}}$, we now have three equations for how $\sigma_{x}^2$, $\sigma_{y}^2$, and $C_\x{f}$ change over time:
%\begin{align*}
%\sigma_{x}^2(t+1)&=\frac{\sigma_x^2}{4}\left(\frac{\sigma^2(\sigma^2+\sigma_{x}^2)+\sigma_{x}^2\sigma_{y}^2}{(\sigma^2+\sigma_{x}^2)^2}+\frac{2C_\x{f}}{\sigma^2+\sigma_x^2}+1\right)
%\\\sigma_{y}^2(t+1)& =\sigma_{x}^2\left(\frac{\sigma^2(\sigma^2+\sigma_{x}^2)+\sigma_{x}^2\sigma_{y}^2}{(\sigma^2+\sigma_{x}^2)^2}\right)
%\\C_\x{f}(t+1)&=\frac{1}{2}\sigma_{x}^2\left(\frac{\sigma^2(\sigma^2+\sigma_{x}^2)+\sigma_{x}^2\sigma_{y}^2}{(\sigma^2+\sigma_{x}^2)^2}\right)+\frac{1}{2}\sigma_{x}^2\frac{C_\x{f}}{\sigma^2+\sigma_{x}^2}.
%\end{align*}
In the Supporting Information, we show that there are two stable equilibria of these dynamics. The first occurs at $\sigma_x^{2\star}=C_\x{f}^\star=0$. The second stable equilibrium occurs when
\begin{align*}
\sigma_x^{2\star}&=\frac{3\sigma_y^2-5\sigma^2+\sqrt{9(\sigma_y^2)^2-30\sigma^2\sigma_y^2+(\sigma^2)^2}}{6}
\\ C_\x{f}^\star&=\frac{\sigma_x^{2\star}\sigma_y^2}{\sigma^2+\sigma_x^{2\star}}.
\end{align*}
In the Supporting Information, we derive the corresponding recursion equations describing how $\sigma_x^2$, $\sigma_y^2$, and $C$ change.

\section*{Results}
\subsection*{Variance of song distribution at equilibrium. }
The stable equilibria of all nine modes of acquisition are presented in Table \ref{equilibrium}. We will first focus on the equilibrium values of $\sigma_x^2$. As noted above, when song is learned obliquely from a randomly chosen adult the variance of the distribution of songs does not change, so its equilibrium value $\sigma_x^{2\star}$ is equal to its initial value $\sigma_x^2(0)$. Of the other six modes, there are three in which $\sigma_x^2$ can approach a non-zero equilibrium and three in which $\sigma_x^2$ always approaches $0$.  If $\sigma_x^2>0$ at equilibrium, multiple songs are present in the population. On the other hand, if $\sigma_x^2$ approaches $0$, only one song is present in the population and there is no diversity.

\begin{enumerate}
\item
In mode $2$ (song is genetic and preference is maternally learned), the system of recursion equations for $\sigma_x^2$ and $C$ is bistable ($\sigma_y^2$ does not change). There are two stable equilibria: one where $\sigma_x^{2\star}=C=0$ and another where $\sigma_x^{2\star}>0$, as long as $\sigma_y^2\geq\frac{5+2\sqrt{6}}{3}\sigma^2$. Otherwise $\sigma_x^{2\star}=C=0$ is the only stable equilibrium. If the initial values of $\sigma_x^2$ and $C$ are small, the dynamics will reach the first equilibrium, and if they are high enough, the dynamics will reach the second equilibrium. See Figure \ref{mode9} for an example of trajectories that lead to each of these equilibria. 
\item In mode $3$ (song is paternally learned and preference is maternally learned), the recursion equations of $\sigma_x^2$ and $C$ only have one stable equilibrium, $\sigma_x^{2\star}=\max\{\sigma_y^2-\sigma^2,0\}$, which is greater than $0$ as long as $\sigma_y^2>\sigma^2$.
\item In mode $5$ (song is genetic and preference is genetic), the system of recursion equations for $\sigma_x^2$, $\sigma_y^2$, and $C$ is bistable. There is a stable equilibria where $\sigma_x^{2\star}=\sigma_y^{2\star}=C=0$, which is reached if the initial values of $\sigma_x^2$, $\sigma_y^2$, and $C$ are small. If the initial values $\sigma_x^2$, $\sigma_y^2$, and $C$ are large enough, all three will increase indefinitely. (See Figure \ref{mode7}.)
\item In modes $6$, $8$, and $9$ the only stable equilibrium occurs when $\sigma_x^{2\star}=0$. (See Figures \ref{mode8}, \ref{mode1}, and \ref{mode2} respectively). 
\end{enumerate}

For both traits, there are the most modes in which the variance of songs at equilibrium is greater than $0$ when the trait is maternally or obliquely learned, the second most when the trait is genetic, and the least when the trait is paternally learned. The rows and columns of Table \ref{equilibrium} are ordered so that $\sigma_x^{2\star}=0$ in cells that are lower and to the right. When song is obliquely learned, $\sigma_x^{2\star}=\sigma_x^2(0)>0$, regardless of how preference is acquired. In other words, oblique learning of song permits the variation of songs present in the population to persist undiminished through the generations. Similarly, when preference is maternally learned, song variation can persist ($\sigma_x^{2\star}>0$), regardless of how song is acquired.  When song is genetic, both maternal inheritance and genetic inheritance of preference permit persistence of song variation. When preference is genetic, both oblique learning and genetic inheritance of song permit persistence of song variation.  When song is paternally learned, only maternal inheritance of preference permits persistence of song variation. When preference is paternally learned, only oblique song learning permits persistence of song variation.  

Conversely, song variance tends to be greater when song is paternally learned than when song is genetic. This is true asymptotically for the modes and parameters where the equilibrial variance of songs is positive (compare the light red and blue curves Figure \ref{sigmax2_sigma2}A). When preference is learned either maternally or paternally, it is true transiently for the modes and parameters where the equilibrial variance is zero,  (compare the dark red and blues curves in Figure \ref{sigmax2_sigma2}A and C). When preference is genetic, the transient level of song variance can be slightly higher when song is genetic than when it is paternally learned (compare the dark red and blue curves in Figure \ref{sigmax2_sigma2} B), but even then it takes more generations for song variance to approach zero when song is paternally learned (compare the dark red and blue curves in Figure \ref{sigmax2_sigma2}E).


%Similarly, when preference is learned from a female's mother, the equilibrium value $\sigma_y^{2\star}=\sigma_y^2(0)$. 

\subsection*{Female choosiness. }
Both equilibrium and transient variance are highest at small values of $\sigma^2$. Equilibrium song variance ($\sigma_x^{2\star})$ decreases monotonically as a function of $\sigma^2$ (light red and blue curves in Figure \ref{sigmax2_sigma2}A). (We derive the derivatives of $\sigma_x^{2\star}$ for these modes and show that they are in fact negative in the Supporting Information.) When song is genetic, transient song variance ($\sigma_x^2(10)$) decreases monotonically as a function of $\sigma^2$ (dark red curves in Figure \ref{sigmax2_sigma2}A-C). When song is paternally learned, transient song variance does not strictly decrease with $\sigma^2$, but it is highest at low values of $\sigma^2$ (dark blue curves in Figure \ref{sigmax2_sigma2}A-C).  (We show $\sigma_x^{2\star}$, $\sigma_x^2(10)$, $\sigma_y^{2\star}$, and $\sigma_y^2(10)$ as a function of $\sigma^2$, as well as of initial values $\sigma_x^2(0)$ and $\sigma_y^2(0)$, in Figures \ref{sigmax2_full} and \ref{sigmay2_full}.)
%Finally, we measure the number of generations it takes before $\sigma_x^2$ is less than a threshold of $5\times 10^{-3}$. For those cases where song is genetic ($2$, $5$ and $8$), this number decreases as a function of $\sigma^2$ (Figure \ref{sigmax2_sigma2}). On the other hand, for those cases where song is paternally learned ($6$ and $9$), this number tends to increase as a function of $\sigma^2$. 

\subsection*{Alternate equilibria. }
Until this point, we have assumed that both male song and female preference are initially normally distributed. With the addition of a Gaussian preference function, this ensures that both traits continue to be normally distributed across generations. Here we explore how the dynamics given by the mating rules in Equation \ref{model} are affected when we use a non-normal distribution either trait or a non-Gaussian preference function. In these cases, it is not possible to write down an analytical expression for the distribution of traits in the offspring generation. Instead, it becomes necessary to find the distribution of traits among offspring numerically. To do so, we restrict both traits to be within a range $[-M,M]$ and we establish a partition of this range with an step size of $\delta$:
$$
S=\{-M,-M+\delta,\dots,-2\delta,-\delta,0,\delta,2\delta,\dots,M-\delta,M\}.$$
For example, if $M=10$ and $\delta=0.01$, we consider a partition $$S=\{-10,-9.99,\dots,-0.02,-0.01,0,0.01,0.02,\dots,9.99,10\}$$
We only consider trait values that are in this partition, for both songs and preferences. For adults of each sex, we then construct a probability distribution over trait values in this partition:
\begin{align*}
P_\x{m}(v) &\text{ such that } \sum_{x\in S}\sum_{y \in S}P(x,y)=1
\\ \text{ and } P_\x{f}(v) &\text{ such that } \sum_{x\in S}\sum_{y \in S}P_\x{f}(x,y)=1
\end{align*}
We also have to define a preference function $f_y$ over this partition, such that $\sum_{x\in S}f_y(x)=1$. Given initial trait distributions and a preference function, we use a slightly modified version of Equation \ref{model} to describe the probability of a particular pair mating. If $u=(x_\x{m},y_\x{m},x_\x{f},y_\x{f})^T$, the probability of a pair with traits $(x_\x{m},y_\x{m})$ and $(x_\x{f},y_\x{f})$ mating is 
\begin{equation}
P_\text{mate}(u)=\frac{P_\x{f}(v_\x{f})P_\x{m}(v_\x{m})f_{y_\x{f}}(x_\x{m})}{Z_{y_\x{f}}}, \label{model_numerical}
\end{equation}
where $Z_{y_\x{f}}=\sum_{x\in S}\sum_{y\in S}P_\x{m}((x,y))f_{y_\x{f}}(x_\x{m})$. We can find this distribution numerically and then sum over the appropriate dimensions of $P_\text{mate}(u)$ to find the distribution of each trait in the offspring generation. We find such distributions for each of the nine modes of acquisition.

We can use this scheme with the same normal trait distributions and Gaussian preference functions as we used above, with the slight modification that we have to normalize the distributions and preference function. For example, instead of using 
\begin{align*}
P_\x{m}(v)=\frac{1}{2\pi\sqrt{|\Sigma_\x{m}|}}\exp\left(-\frac{1}{2}(v-\mu_\x{m})^T\Sigma_\x{m}^{-1}(v-\mu_\x{m})\right)
\end{align*}
for any $v\in\R^2$, we use
\begin{align*}
P_\x{m}(v)=\frac{\frac{1}{2\pi\sqrt{|\Sigma_\x{m}|}}\exp\left(-\frac{1}{2}(v-\mu_\x{m})^T\Sigma_\x{m}^{-1}(v-\mu_\x{m})\right)}{\sum_x\sum_y\frac{1}{2\pi\sqrt{|\Sigma_\x{m}|}}\exp\left(-\frac{1}{2}(v-\mu_\x{m})^T\Sigma_\x{m}^{-1}(v-\mu_\x{m})\right)}
\end{align*}
for $v=\in S\times S$. We can normalize a normal distribution of preferences and a Gaussian preference function similarly. 

Our purpose in using this numerical process to see how the distribution of the traits change over time is to explore the robustness of the previous results to using different initial trait distributions and a different preference function. In particular, we will use step functions for each of these.  
%%% took a bunch out for the supp info but do need to define $\mu$, $p_1$, and $p_2$: We will then construct a step function $P(x)$ over $S$ such that $P(x)=p_1$ for $x\in S_1$ or $S_3$ and $P(x)=p_2$ for $x\in S_3$.
For these analyses, we focus on modes of acquisition $3$ and $7$. In mode $3$ preference is maternally learned and the variance in the distribution of songs reaches a nonzero equilibrium, whereas in mode $7$ song is learned obliquely and the variance in the distribution of preferences reaches a nonzero equilibrium. In each, the variance of the trait of interest (song in the former and preference in the latter) does not depend on the shape of the preference being used (Figure \ref{effect_of_step_function}). In other words, the results we showed above are robust to using different preference functions.

Using initial distributions of each trait also do not strongly affect the variance of the traits of interest in these modes (Figure \ref{effect_of_step_dist}). Using a step function as the distribution of songs does change the equilibrium variance of the preference distribution in mode $7$ a little. However, the trend in this variance as a function of the variance of the preference function is very similar. Otherwise, regardless of the widths of the steps used in the step function for either trait distribution, the equilibrium variance of the song distribution in mode $3$ and the equilibrium variance of the preference distribution in mode $7$ are almost exactly the same as if the initial conditions involved normal distributions. 

Up to now, we have been focusing on the variance of the trait distributions at equilibrium because they have been unimodally distributed and variance is a convenient was of measuring the shape of a unimodal distribution. However, when we use step functions for the initial distribution of female preferences, the distribution of songs in mode $3$ does not approach a unimodal distribution. Rather, the equilibrium distribution exhibits multiple peaks (Figure \ref{effect_of_step_dist}). 

These peaks occur because of how the songs a female finds attractive are distributed. Remember that $Z_{y_\x{f}}=\sum_{x\in S}\sum_{y\in S}P_\x{m}((x,y))f_{y_\x{f}}(x_\x{m})$ indicates the total fraction of males a female with preference $y$ finds attractive. When the distribution of male songs follows a step function, rather than a normal distribution, there are somewhat more males singing the ``average" song and and more singing extreme songs, but fewer males singing songs in between. %KURTOSIS?? XXX
Consequently, females that prefer average songs and extreme songs have many options, but females that prefer songs in between have fewer options. The desperation of females that prefer intermediate songs means that males who sing those songs have disproportionate mating success. Intermediate males increase in frequency until there are enough of them to sate the demand of previously desperate females. As this peak is established, there are troughs on either side that are then favored because the females that prefer those songs do not have many options. This process of peaks appearing continues until the distribution reaches an equilibrium, where the desperation of females is balanced XXXX

\section*{Discussion}

%There are two cases in which song is genetic that can maintain song diversity ($2$ and $5$), but only one in which song is paternally learned than can maintain song diversity ($3$). On the other hand, there is a higher amount of song diversity present in those cases where song is paternally learned compared to those cases where song is genetic (compare mode $3$ to $2$, $6$ to $5$, and $9$ to $8$). This leads to the following prediction: of those species that have multiple song types, more should have genetic song than paternally learned song, but there should be more song types in species in which song is paternally learned than in species in which song is genetic. 

%When we use initially normally distributed traits, the traits continue to be normally distributed until they reach equilibrium. When we use traits that are initially distributed according to a step function, in some cases we find distributions of songs at equilibrium that exhibit multiple peaks. Rather than interpreting the presence of multiple discrete song types as a transient condition that will eventually be replaced by one in which only one song is present, it is possible that such a pattern can be maintained even by the simple mating process we model here. While we found this pattern by using initial distributions that followed step functions, other initial conditions could lead to the same patterns. The important thing is not the exact shape of the initial conditions, since the step function shape disappears after only a few generations. 


\newpage

\bibliographystyle{plainnat}
\bibliography{song_learning_evolution}

\begin{table}
\caption{\label{variables} Table of variables used in the text. In this table, we only give the version of each variable with a male-specific subscript for brevity. For example, while we use both $\sigma_{x\x{m}}^2$ and $\sigma_{x\x{f}}^2$ in the text, we only include $\sigma_{x\x{m}}^2$ in the table.}
\begin{tabular}{lllll}
Variable & Interpretation
\\\hline $C_\x{m}=\rho_\x{m}\sigma_{x\x{m}}\sigma_{y\x{m}}$ & covariance of traits among adult males
\\ $C^\star$ & equilibrium covariance of traits
\\ $\mu$ & mutation rate 
\\$\mu_\x{m}=(\mu_{x\x{m}},\mu_{y\x{m}})^T$ & vector of average values of each trait among adult males 
\\$m,n$ & give widths of step function
\\ $\rho_\x{m}=\frac{C_\x{m}}{\sigma_{x\x{m}}\sigma_{y\x{m}}}$ & correlation between traits among adult males
\\$\Sigma_\x{m}$ & covariance matrix of traits among adult males
\\$\Sigma_\text{mate}$ & covariance matrix of traits among mating adults
\\$\sigma_{x\x{m}}^2$ & variance of songs among adult males
\\$\sigma_{x}^{2\star}$ & equilibrium variance of songs
\\$\sigma_{y\x{m}}^2$ & variance of preferences among adult males
\\$\sigma_y^{2\star}$ & equilibrium variance of preferences
\\$\sigma^2$ & variance of preference function
\\$u=(x_\x{m},y_\x{m},x_\x{f},y_\x{f})^T$ & vector of both traits in both sexes
\\$v=(x,y)^T$ & vector of traits
\\$x$ & song
\\$y$ & preference
\\$Z_y=\int_{\R^2}P_\x{m}(v_\x{m})f_{y}(x_\x{m})dv_\x{m}$ & integral of males a female with preference $y$ finds attractive
\\ $\delta$ & width of steps of the partition $S$
\\ $M$ & half-width of interval over which we find probability distribution numerically
\\ $S$ & partition that includes the discrete set of traits we consider in our numerical analyses
\end{tabular}
\end{table}


\begin{table}
\caption{\label{equilibrium}Here we show the stable equilibria of the nine possible modes of acquisition. Stable equilibria are indicated with a ${}^\star$ and initial values are indicated with $(0)$. For example $\sigma_x^{2\star}$ is a stable equilibrium of $\sigma_x^2$, and $\sigma_x^2(0)$ is the initial value of $\sigma_x^2$. There are two modes in which the system of recursion equations for $\sigma_x^2$ and $C$ are bistable: when song is genetic and preference is maternally learned and when both song and preference are genetic. Both stable equilibria are included in the table. The second equilibrium when song is genetic and preference is maternally learned only exists when $\sigma_y^2\geq\frac{5+2\sqrt{6}}{3}\sigma^2$. The rows and columns are ordered such that the equilibrium value $\sigma_x^{2\star}$ is smaller in cells that are lower and to the right. }
\begin{tabular}{|l|l|l|l|}
\hline & A. song learned obliquely  & B. song genetic & C. song learned from father
\\\hline I. pref from mother  & $\sigma_x^{2\star}=\sigma_x^2(0)$ & $\sigma_x^{2\star}=0, \ \frac{3\sigma_y^2-5\sigma^2+\sqrt{9(\sigma_y^2)^2-30\sigma^2\sigma_y^2+(\sigma^2)^2}}{6}$ & $\sigma_x^{2\star}=\max\{\sigma_y^2-\sigma^2,0\}$  
\\ 	& 	$\sigma_y^{2\star}=\sigma_y^2(0)$ 	& $\sigma_y^{2\star}=\sigma_y^2(0)$ 		  & $\sigma_y^{2\star}=\sigma_y^2(0)$   
\\ & $ C^\star=0$ &   $ C^\star=\frac{\sigma_x^{2\star}\sigma_y^{2\star}}{\sigma^2+\sigma_x^{2\star}}$  & $ C^\star=0$
\\\hline II. pref genetic &  $\sigma_x^{2\star}=\sigma_x^2(0)$  & $\sigma_x^{2\star}=0,\ \infty$  & $\sigma_x^{2\star}=0$                      
\\  		&  $\sigma_y^{2\star}=0$	& $\sigma_y^{2\star}= 0 , \ \infty$ 	  & $\sigma_y^{2\star}=0$  
\\ & $ C^\star=0$   & $ C^\star=0, \ \infty$        & $ C^\star=0$          
\\\hline III. pref from father & $\sigma_x^{2\star}=\sigma_x^2(0)$ & $\sigma_x^{2\star}=0$  & $\sigma_x^{2\star}=0$                       
\\  			& $\sigma_y^{2\star}=\frac{\sigma_x^2(\sigma^2+\sigma_x^2)}{2\sigma_x^2+\sigma^2}$	  & $\sigma_y^{2\star}=0$  & $\sigma_y^{2\star}=0$                       
\\ & $ C^\star=0$ & $ C^\star=0$ & $ C^\star=0$
\\\hline
\end{tabular}
\end{table}

\begin{figure}
\includegraphics[width=6.5in]{/Users/eleanorbrush/Desktop/sigmax2_by_female_mode.pdf}
\caption{\label{sigmax2_sigma2} Song variance is highest when females are choosy. In the left column, we show either equilibrium variance ($\sigma_x^{2\star}$) or transient variance ($\sigma_x^2(10)$) as a function of the variance of the preference function ($\sigma^2$). In the right column, we show the number of generations before $\sigma_x^2$ is less than $0.05$, for those cases where $\sigma_x^{2\star}=0$. In the top row, preference is maternally learned. In the second row, preference is genetic. In the bottom row, preference is paternally learned. Red indicates song is genetic; blue indiciates song is paternally learned. In this figure we ignore the modes in which song is learned obliquely. We use lighter colors for parameters where $\sigma_x^{2\star}>0$ and darker colors for parameters where $\sigma_x^{2\star}=0$. In the middle row, B and E, the dark red curves (meaning both song and preference are genetic) start at $\sigma^2\approx 1.3$ because, at smaller values of $\sigma^2$, $\sigma_x^{2\star}=\infty$. Parameters: $\sigma_x^2(0)=0.8$, $\sigma_y^2(0)=4$, $\rho(0)=0.6$. In this figure traits are normally distributed and we use a Gaussian preference function. }
\end{figure}

\begin{figure}
\includegraphics[width=6.5in]{/Users/eleanorbrush/Desktop/effect_of_step_function.pdf}
\caption{\label{effect_of_step_function} Using a step preference function, instead of a Gaussian preference function, has barely any effect on the equilibrium distributions of song and preference. In the upper row, we show examples of preference functions that have the same variance, in the left panel $\sigma^2=0.3$ and in the right $\sigma^2=0.9$. The red curve in each panel is a Gaussian with the variance given. Curves with the same color in the two panels have steps of the same width. There is no purple function in the left panel because no function with those widths can have the desired variance. In the lower left panel, we show how the equilibrium variance of the song distribution in mode $3$ (song is paternally learned and preference is maternally learned) as a function of $\sigma^2$. The colors of the curves correspond to the colors of the preference functions in the upper panels. The curves overlap, indicating that the shape of the preference function does not affect the variance of the equilibrium distribution of songs. In the lower right panel, we show how the equilibrium variance of the preference distribution in mode $7$ (song is learned obliquely and preference is paternally learned) as a function of $\sigma^2$. The colors of the curves correspond to the colors of the preference functions in the upper panels. There are differences between the curves, but they are overall very similar to each other, indicating that the shape of the preference function does not have a large effect on the variance of the equilibrium distribution of preferences. Parameters: $\mu=0.01$, $\rho(0)=0$, $\sigma_x^2(0)=0.8$, $\sigma_y^2(0)=2$, number of generations = $5000$. In this figure, both songs and preferences are initially normally distributed.}
\end{figure}

\begin{figure}
\includegraphics[width=6.5in]{/Users/eleanorbrush/Desktop/effect_of_step_distribution.pdf}
\caption{\label{effect_of_step_dist}The model allows for equilibrium distributions of songs that are multimodal. In this figure, we show the effect of using step functions as initial conditions for songs and preferences, while using a Gaussian preference function. In the upper row, we show the step functions that we use as initial distributions of songs on the left and of preferences on the right. The red curve in each panel is a normal distribution. 
%All of the distributions on the left have variance $\sigma_x^2(0)=0.8$ and all of the distributions on the right have variance $\sigma_y^2(0)=2$. 
In the left column, we use the step functions in the top left panel for the initial distributions of songs, while using a normal distribution of preferences. In the right column, we use the step functions in the top right panel for the initial distributions of preferences, while using a normal distribution of songs.  In the second row, we show the equilibrium variance of the song distribution in mode $3$ (song is paternally learned and preference is maternally learned) as a function of the variance of the preference function ($\sigma^2$). In the third row, we show the equilibrium variance of the preference distribution in mode $7$ (song is learned obliquely and preference is paternally learned) as a function of the variance of the preference function ($\sigma^2$). In the bottom row, we show equilibrium distributions of songs in mode $3$, using step functions as initial conditions of the song distribution on the left and step functions as initial conditions of the preference distribution on the right. Parameters: $\mu=0.01$, $\rho(0)=0$, $\sigma_x^2(0)=0.8$, $\sigma_y^2(0)=2$, $\sigma^2=1.1$ unless it is being varied, number of generations = $5000$. }
\end{figure}

\begin{figure}
\includegraphics[width=6.5in]{/Users/eleanorbrush/Desktop/peak_example.pdf}
\caption{\label{peak_example}  A multimodal distribution occurs when the distribution of $Z$, the total fraction of males available to females with preference $y$, diverges from a normal distribution. In A we show the initial distribution of preferences in the population, red indicating a normal distribution and black a step function. In B we show the equilibrium distribution of songs following from those initial conditions. In C we show $Z$ at $10$ %XXX may change---keep track of parameters!
generations as a function of female preference. In D we show the difference in the two curves in C, the black curve minus the red curve, as a function of female preference. The peaks in this curve lead to peaks in the distribution of songs. Parameters: $\mu=0.01$, $\rho(0)=0$, $\sigma_x^2(0)=0.8$, $\sigma_y^2(0)=2$, $\sigma^2=0.9$, number of generations = $5000$. In this figure, we use a Gaussian preference function. 
}
\end{figure}

\clearpage{}
\renewcommand{\thesection}{}
\renewcommand{\thesection}{S}
\renewcommand{\thesubsection}{S\arabic{subsection}}
\renewcommand{\theequation}{S\arabic{equation}}
\renewcommand{\thetable}{S\arabic{table}}
\renewcommand{\thefigure}{S\arabic{figure}}
\setcounter{equation}{0}  
\setcounter{figure}{0}
\setcounter{table}{0}

\begin{figure}
\includegraphics[width=3.25in]{/Users/eleanorbrush/Desktop/mode9_dynamics}
\caption{\label{mode9}}
\end{figure}

\begin{figure}
\includegraphics[width=3.25in]{/Users/eleanorbrush/Desktop/mode7_dynamics}
\caption{\label{mode7}}
\end{figure}

\begin{figure}
\includegraphics[width=3.25in]{/Users/eleanorbrush/Desktop/mode8_dynamics}
\caption{\label{mode8}}
\end{figure}

\begin{figure}
\includegraphics[width=3.25in]{/Users/eleanorbrush/Desktop/mode1_dynamics}
\caption{\label{mode1}}
\end{figure}


\begin{figure}
\includegraphics[width=3.25in]{/Users/eleanorbrush/Desktop/mode2_dynamics}
\caption{\label{mode2}}
\end{figure}


\begin{figure}
\includegraphics[width=6.5in]{/Users/eleanorbrush/Desktop/sigmax2_full.pdf}
\caption{\label{sigmax2_full} The variance of the song trait $\sigma_x^{2}$ increases as a function of $\sigma_y^2(0)$ and decreases or is non-monotonic as a function of $\sigma^2$. There are two possibilites for the dynamics of $\sigma_x^2$: the equilibrium value of $\sigma_x^2$ can either be greater than $0$ or equal to $0$. In this figure we ignore the modes in which song is learned obliquely. In the top row, we show $\sigma_x^{2\star}$ for two other modes in which $\sigma_x^{2\star}>0$ ($2$ and $3$). (In mode $5$, $\sigma_x^2$ approaches infinity, which we do not show here.) In the second row, we show $\sigma_x^2(10)$  for those modes in which $\sigma_x^{2\star}=0$ ($2$, $5$, $6$, $8$, and $9$). In the third row, we show the number of generations before $\sigma_x^2$ is less than $5\times10^{-3}$ for those modes in which $\sigma_x^{2\star}=0$. In the first column we show these quantities as a function of initial value $\sigma_x^{2}(0)$. In the second column we show these quantities as a function of initial value $\sigma_y^2(0)$. In the third column we show these quantities as a function of $\sigma^2$. Note that for the bistable modes, $2$ and $5$, some parameters give $\sigma_x^{2\star}>0$ and others give $\sigma_x^{2\star}=0$, i.e. the grey curves is split between the top and middle rows and the green curve only appears for some parameters in the middle row. Parameters: unless the parameter is being varied $\sigma_x^2(0)=0.8$, $\sigma_y^2(0)=4$, $\sigma^2=1$, $\rho(0)=0.6$. In this figure both songs and preferences are initially normally distributed and we use a Gaussian preference function. } 
\end{figure}

\begin{figure}
\includegraphics[width=6.5in]{/Users/eleanorbrush/Desktop/sigmay2_full.pdf}
\caption{\label{sigmay2_full} The variance of the preference trait $\sigma_y^{2}$ increases as a function of $\sigma_y^2(0)$ and decreases or is non-monotonic as a function of $\sigma^2$. There are two possibilites for the dynamics of $\sigma_y^2$: the equilibrium value of $\sigma_y^2$ can either be greater than $0$ or equal to $0$. In this figure we ignore the modes in which preference is maternally learned. In the top row, we show $\sigma_y^{2\star}$ for the other mode in which $\sigma_y^{2\star}>0$ ($7$). (In mode $5$, $\sigma_y^2$ approaches infinity, which we do not show here.) In the second row, we show $\sigma_y^2(10)$  for those modes in which $\sigma_y^{2\star}=0$ ($4$, $5$, $6$, $8$, and $9$). In the third row, we show the number of generations before $\sigma_y^2$ is less than $5\times10^{-3}$ for those modes in which $\sigma_x^{2\star}=0$. In the first column we show these quantities as a function of initial value $\sigma_x^{2}(0)$. In the second column we show these quantities as a function of initial value $\sigma_y^2(0)$. In the third column we show these quantities as a function of $\sigma^2$. Note that for the bistable mode, $5$, some parameters give $\sigma_y^{2\star}>0$ and others give $\sigma_y^{2\star}=0$, i.e. the green curve only appears for some parameters in the middle row.  Parameters: unless the parameter is being varied $\sigma_x^2(0)=0.8$, $\sigma_y^2(0)=4$, $\sigma^2=1$, $\rho(0)=0.6$. In this figure both songs and preferences are initially normally distributed and we use a Gaussian preference function.}
\end{figure}



\end{document}
