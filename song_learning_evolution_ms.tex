\documentclass[12pt]{article}
\usepackage[sc]{mathpazo} %Like Palatino with extensive math support
\usepackage{fullpage}
\usepackage{setspace,lipsum}
%\usepackage[textwidth=7in,textheight=8in]{geometry}
%\usepackage[authoryear,sectionbib]{natbib}
\usepackage[numbers,sort&compress]{natbib}
\linespread{1.7}
\usepackage[utf8]{inputenc}
\usepackage{latexsym}
\usepackage{amssymb,amsmath}
\usepackage{graphicx}
\usepackage{/Users/eleanorbrush/Documents/custom2}
\newcommand{\ra}[1]{\renewcommand{\arraystretch}{#1}}
\newcommand{\x}[1]{\text{#1}}
\newcommand{\Cov}{\text{Cov}}
\usepackage{lscape} 
\newcommand\numberthis{\addtocounter{equation}{1}\tag{\theequation}}
\usepackage{lineno}
\usepackage{enumerate}
\usepackage{multirow}
\linenumbers{}
%\modulolinenumbers[3]

\usepackage{xr}
\externaldocument{supp_info}

\usepackage{fancyhdr}
\setlength{\headheight}{15pt}
\pagestyle{fancy}
%\lhead{The Effect of Song Learning Depends on How Female Preferences Are Acquired}

\setlength{\headsep}{0.15in}

\title{The Effect of Song Learning Depends on How Female Preferences Are Acquired}

\author{Eleanor Brush$^{1\ast}$, William F. Fagan$^{1}$}


\date{}

\begin{document}

\maketitle

\noindent{} 1. Department of Biology, University of Maryland;

\noindent{} $\ast$ Corresponding author; e-mail: eleanor.brush@gmail.com.


\bigskip

\textit{Keywords}: learning, song, evolution, variation, quantitative genetics, sexual selection.

\bigskip

\bigskip

%\linenumbers{}
%\modulolinenumbers[3]
\thispagestyle{fancy}
\newpage{}

\section*{Abstract}

\section*{Introduction}

The basic principle of Darwinian evolution---that the frequencies of favored traits in a population increase over time---apply regardless of how a trait is acquired. In fact, mathematical models that describe the cultural evolution of learned traits and those that describe the biological evolution of genetic traits are very similar \cite{Mesoudi:2006fk}. However, there are fundamental differences between traits that are learned and those that are genetically inherited: animals that can learn can be more plastic and flexible than those whose behaviors are entirely genetic \cite{Ancel:2000vn,Anderson:1995ys}. This plasticity reduces the correlation between genotype and phenotype because animals with different genotypes can learn the same behavior. Sometimes this plasticity slows down the rate at which natural selection drives a population towards an optimal genotype \cite{Lachlan:2004tg,Ancel:2000vn,Anderson:1995ys}, but under other circumstances, it accelerates the rate of evolution \cite{Ancel:2000vn}. Recognizing this link between the mode of trait inheritance and evolutionary dynamics is important for understanding the similarities and differences between cultural and biological evolution. In this paper, we focus on one particular trait that can be either learned or genetically inherited: bird song.  

In many species of birds, females make mating choices based on the songs that males sing. If two subpopulations differ in the songs they produce, they may therefore become reproductively isolated and speciation can occur \cite{Slabbekoorn:2002kl}. If learning generates variation more quickly than genetic inheritance, then those species that learn their songs could speciate more quickly \cite{Beecher:2005ly}. One compelling piece of evidence supporting this argument is the fact that, of the roughly forty orders of birds, one of the three that contains birds who learn their songs accounts for more than half of the known species of birds. While the astounding overrepresentation of this order may be due to the fact that they learn their songs, not all studies have found the predicted relationship between speciation rates and learning (reviewed in \cite{Wilkins:2012ve,Lachlan:2004tg}).  Some theoretical models and thought experiments have shown that male songs that are learned \cite{Lachlan:2004tg,Irwin:2012hc} or females preferences that are learned \cite{Gilman:2015fk,Servedio:2013uq,Bailey:2012kx,Irwin:1999fk} can accelerate speciation. The opposite argument---that learning may prevent speciation---has also been proposed: if birds from diverging populations can learn each others' songs, then they may interbreed more readily and reproductive isolation may be prevented \cite{Seddon:2008bh}.  A full understanding of how bird song learning facilitates or hinders speciation requires understanding the interaction between multiple diverging subpopulations. As a first step, we must understand how much variation in bird song can be supported in a single population and how this depends on how song is acquired. Overall, theoretical models of the evolution of bird song have led to conflicting findings about the effects of learning \cite{Verzijden:2012uq}. 


Among birds that learn their songs, there are many strategies for doing so  \cite{Beecher:2005ly}. 
%In the most speciose order of all (the Passeriformes), there is evidence for learned songs primarily in one suborder (the oscine passerines, commonly referred to as the songbirds) \cite{Seddon:2008bh}, with possibly one exception \cite{Saranathan:2007dq}.  
In particular, males can learn from different sources. In some species, males learn their songs from their fathers (for example, in Darwin's finches \cite{Grant:1996ve}), which is called vertical learning \cite{Cavalli-Sforza:1981bc}. In others, males learn from an unrelated male (for example, in white-crowned sparrows \cite{Baptista:1988qf}), which is called oblique learning \cite{Cavalli-Sforza:1981bc}. Even when song is learned, there can be genetic contributions \cite{Forstmeier:2009cr,Slabbekoorn:2002kl,Airey:2000dq}, as with any complex behavior.
Females can also acquire their preferences in different ways. In some species that female preferences are genetic \cite{Verzijden:2005vn}. In others, female preferences seem to be learned. In this case, a female's preference can be learned by imprinting on a father's song \cite{Verzijden:2005vn} or by copying other females' choices \cite{Kirkpatrick:1994vn}. 
%\begin{itemize}
%\item examples of females imprinting on males \cite{Verzijden:2005vn}
%\item examples of genetic preference   \cite{Verzijden:2005vn}
%\item females learn preferences by copying other females \cite{Kirkpatrick:1994vn}
%\item mathematical model in which females learn to be choosy by copying their mothers, though no examples given \cite{Laland:1994ys}
%\item size of brain region associated with song complexity is heritable in zebra finches \cite{Airey:2000dq}
%\item some of the variation in both female preferences for songs and male songs in the field cricket can be explained by genetics \cite{Gray:1999dz}
%\end{itemize}


The ``model" or ``imprinting set" upon which young birds base either their songs or their preferences helps determine the evolutionary dynamics of those traits \cite{Tramm:2008ij}. If we consider songs that can be paternally learned, obliquely learned, or genetically inherited, and preferences that can be maternally learned, paternally imprinted, obliquely imprinted, or genetically inherited, this gives rise to twelve possible combinations of how the two traits are acquired. However, in most previous studies one of two simplifying assumptions was made:  either (a) the mechanism by which one trait is acquired was fixed and the authors studied the effects of varying how the other trait is acquired (for example in \cite{Gilman:2015fk,Verzijden:2005vn,Kirkpatrick:1994vn,Lachlan:2004tg}) or (b) it is assumed that the two traits are acquired in the same way and the authors study the effect of varying this one mechanism of acquisition (for example in \cite{Yeh:2015bh}). 
%--- song genetic, preference genetic or learned \cite{Verzijden:2005vn}
%--- song from father (haploid or Y-linked), preference learned from older females  either en masse or by choosing an individual to copy \cite{Kirkpatrick:1994vn}
%--- ecological phenotype is genetic, preference is genetic, maternally imprinted, paternally imprinted, or obliquely imprinted \cite{Gilman:2015fk}
%--- preference is genetic, song has genetic component but is either learned or genetic \cite{Lachlan:2004tg}
%--- both males and females have a trait, females mate assortatively, trait can be genetic, maternally, obliquely, or paternally learned \cite{Yeh:2015bh}
No previous studies, to our knowledge, have compared the evolutionary dynamics of all twelve possible mechanisms.

%The evolution of bird song is a complicated phenomenon for another reason: a bird's song is a complex trait. Not only can members of the same species sing different songs from each other (e.g. in Darwin's finches \cite{Gibbs:1990fk}), but individual males can have large repertoires of songs to choose from  \cite{Devoogd:1993uq,Beecher:2005ly}. 
Song can be thought of as a discrete trait: a male sings either song A or song B.  There are also aspects of song that are better thought of as continuous traits. These can include the average pitch of a song (for example in zebra finches \cite{Forstmeier:2009cr}), its duration (for example in zebra finches \cite{Forstmeier:2009cr}), frequency bandwidth (for example in Darwin's finches \cite{Podos:2001vn}), and trill rate (for example in swamp sparrows \cite{Ballentine:2004kx} and Darwin's finches \cite{Podos:2001vn}). Several studies have focused on the discrete aspect of bird song and compared how the frequencies of discrete song types change over time, depending on whether songs are learned or genetically inherited (e.g. \cite{Lachlan:2004tg,Verzijden:2005vn,Yeh:2015bh}).  While there are some evolutionary models of songs as continuous traits (e.g. \cite{Aoki:2001ly}), few compare the evolutionary dynamics of songs that are learned or genetically inherited. The evolution of continuously distributed traits is described well by quantitative genetics models \cite{Lande:1980zr,Lande:1981fk,Mead:2004uq}, which we use here. 
%Genetic traits that are affected by multiple genes, each of which contributes a small amount to the organism's phenotype, should be continuously distributed \cite{Mead:2004uq,Lande:1981fk,Aoki:2001ly}.


%The likelihood of a population diverging into two species that differ in a particular trait depends on the amount of variation that the population exhibits in that trait.  In order for two subpopulations to speciate based on a particular trait, there needs to be sufficient variation in that trait. On the other hand, if the trait is used in mate choice, if there is too much variation and the ranges of the trait expressed in two diverging subpopulations overlap, then members of the two subpopulations might interbreed more easily, thus preventing speciation. 

In this paper, we study how the variance of the distribution of male songs changes over time in a population where the females use the male's song for mate choice and in which the two traits are continuously distributed. We analyze the dynamics of the distribution of songs for all twelve mechanisms by which the two traits---songs and preferences---can be acquired. 
We find that 
\begin{enumerate}
\item Whether variance in the distribution of songs is maintained depends on how both songs and preferences are acquired. 
\item Variance in the distribution of songs can be maintained at equilibrium when song preferences are maternally learned, song is paternally learned, or when both song and preferences are genetic. The scenario in which both traits are genetic can lead to the variance of both traits increasing indefinitely.
\item The variance in the distribution of songs tends to be higher in those cases when song is paternally learned than in those when song is genetic, except when female preferences are genetic and the initial conditions lead to the variance in songs increasing indefinitely.
\item The variance in the distribution of songs tends to be higher when females are choosy in their mating decisions.
\item The basic mate choice model we study allows for both a normal distribution and a multimodal distribution of songs at equilibrium.
\end{enumerate}

\section*{Model}
\subsection*{Overview. }
\ In our model, we consider a single population of birds. Each male bird has a song it uses to attract female birds. Each female has a preference for a particular song. All females get to mate and have equal mating success. On the other hand, males can mate with multiple females and those with more attractive songs will have higher mating success. Depending on the scenarios we consider, both male song and female preference may be genetically inherited or learned. If song is genetic, then both males and females carry a gene for song, and females may in fact sing, but a bird's song only affects its mating success in males. If song is learned, male birds may learn their songs from their fathers, or they may learn obliquely from a randomly chosen adult male. If preference is genetic, then both males and females carry a gene for preference, but a male bird's preference gene has no effect on who it mates with. If preference is learned, female birds may learn the preferences of their mothers, they may imprint on and prefer the songs of their fathers, or they may imprint on and prefer the song of a randomly chosen adult male. After the adults mate and reproduce, they die and their offspring become next year's adults. In other words, we assume there are non-overlapping generations. Depending on how songs and preferences are acquired, the amount of variation among songs can either increase or decrease over time.

The interesting questions of how female preferences evolved in the first place (for example \cite{Kirkpatrick:1982kl}), how the way in which females learn their preferences evolved (for example in \cite{Tramm:2008ij,Chaffee:2013bs}), and how imitative learning of song evolved \cite{Aoki:1989bh,Olofsson:2008dq} have been addressed elsewhere. Instead, we assume that females \emph{do} have mating preferences and males \emph{are} capable of imitative learning and we then study how the frequencies of females with different preferences and males with different songs change over time.  

\subsection*{Mating. }
\ Mathematically, we will assume every bird has two traits, a song $x$ and preference $y$. (Table \ref{variables} defines all the variables used in the text.) For most of our analyses, we assume that initially both traits are normally distributed among the adults of the population. (We explore different initial conditions below.)
If $v=(x,y)^T$ is the two-dimensional vector giving the song and preference traits, the probability distribution of these traits among all adult males is 
\begin{align*}
P_\x{m}(v)&=\frac{1}{2\pi\sqrt{|\Sigma_\x{m}|}}\exp\left(-\frac{1}{2}(v-\mu_\x{m})^T\Sigma_\x{m}^{-1}(v-\mu_\x{m})\right),
\end{align*} where $\mu_\x{m}=(\mu_{x\x{m}},\mu_{y\x{m}})^T$ gives the expected values of songs and preferences among adult males and 
\begin{align*}
\Sigma_{\x{m}}=\left(\begin{array}{cc}\sigma_{x\x{m}}^2 & \rho_\x{m}\sigma_{x\x{m}}\sigma_{y\x{m}} \\ \rho_\x{m}\sigma_{x\x{m}}\sigma_{y\x{m}} & \sigma_{y\x{m}}^2 \end{array}\right)
\end{align*}
is the covariance matrix for the traits in adult males. Note that the parameter $\rho_\x{m}$ is the correlation between songs and preferences among adult males. We will also use $C_\x{m}=\rho_\x{m}\sigma_{x\x{m}}\sigma_{y\x{m}}$ to denote the covariance between the traits in adult males. Similarly, the probability distribution of these traits among all adult females is 
\begin{align*}
P_\x{f}(v)&=\frac{1}{2\pi\sqrt{|\Sigma_\x{f}|}}\exp\left(-\frac{1}{2}(v-\mu_\x{f})^T\Sigma_\x{f}^{-1}(v-\mu_\x{f})\right), 
\end{align*}
where $\mu_\x{f}$ gives the expected values of songs and preferences among adult females,  $\Sigma_\x{f}$ is the covariance matrix of these traits among adult females, $\rho_\x{f}$ is the correlation between the traits among adult females, and $C_\x{f}$ is the covariance between the traits in adult females. 
% Genetic traits that are affected by multiple genes, each of which contributes a small amount to the organism's phenotype, should be continuously distributed \cite{Mead:2004uq,Lande:1981fk,Aoki:2001ly}
We also assume that each female uses a Gaussian preference function, centered at her preference $y$ with variance $\sigma^2$:
\begin{align*}
f_y(x)&=\frac{1}{\sqrt{2\pi\sigma^2}}\exp\left(\frac{(x-y)^2}{2\sigma^2}\right).
\end{align*}
The variance $\sigma^2$ can be thought of as promiscuity or as inversely related to choosiness since the larger $\sigma^2$ is, the more males a female is attracted to.  (We  discuss different forms for this preference function below.)

The probability of a female with traits $v_\x{f}$ and a male with traits $v_\x{m}$  mating is proportional to the product of the probabilities of finding such a female and such a male, with an additional factor describing the likelihood of the female mating with the male:
\begin{equation} \label{model}
P_\x{mate}(v_\x{m},v_\x{f})=\frac{P_\x{f}(v_\x{f})P_\x{m}(v_\x{m})f_{y_\x{f}}(x_\x{m})}{Z_{y_\x{f}}},
\end{equation}
where $Z_{y_\x{f}}=\int_{\R^2} P_\x{m}(v_\x{m}')f_{y_\x{f}}(x_\x{m}')dv_\x{m}'$ is a normalizing factor that describes the fraction of males a female with preference $y_\x{f}$ finds attractive.  Because we use normal distributions for the initial conditions of both traits and for the preference function, this probability is given by a multivariate normal distribution. Specifically, if $u=(x_\x{m},y_\x{m},x_\x{f},y_\x{f})^T$, under the distribution $P_\text{mate}(u)$, the vector $u$ has expected value 
\begin{align}
\mu_\text{mate}&=\left(\begin{array}{cc} \frac{\sigma^2}{\sigma^2+\sigma_{x\x{m}}^2}\mu_{x\x{m}}+\frac{\sigma_{x\x{m}}^2}{\sigma^2+\sigma_{x\x{m}}^2}\mu_{y\x{f}} 
\\ \mu_{y\x{m}}+\frac{\rho_\x{m}\sigma_{x\x{m}}\sigma_{y\x{m}}}{\sigma^2+\sigma_{x\x{m}}^2}(\mu_{y\x{f}}-\mu_{x\x{m}})
\\ \mu_{x\x{f}}
\\ \mu_{y\x{f}}
 \end{array}\right) \label{mean}
\end{align}
and covariance $\Sigma_\text{mate}=$
\begin{align}
\left(\begin{array}{cccc}\sigma_{x\x{m}}^2\left(\frac{\sigma^2(\sigma^2+\sigma_{x\x{m}}^2)+\sigma_{x\x{m}}^2\sigma_{y\x{f}}^2}{(\sigma^2+\sigma_{x\x{m}}^2)^2}\right) & \frac{\rho_\x{m}\sigma_{x\x{m}}\sigma_{y\x{m}}(\sigma^2(\sigma^2+\sigma_{x\x{m}}^2)+\sigma_{x\x{m}}^2\sigma_{y\x{f}}^2)}{(\sigma^2+\sigma_{x\x{m}}^2)^2} & \sigma_{x\x{m}}^2\frac{\rho_\x{f}\sigma_{x\x{f}}\sigma_{y\x{f}}}{\sigma^2+\sigma_{x\x{m}}^2} &  \sigma_{x\x{m}}^2\frac{\sigma_{y\x{f}}^2}{\sigma^2+\sigma_{x\x{m}}^2}
\\ & \sigma_{y\x{m}}^2\left(\frac{(\sigma^2+\sigma_{x\x{m}}^2)^2-\rho_\x{m}^2\sigma_{x\x{m}}^2(\sigma^2+\sigma_{x\x{m}}^2)+\rho_\x{m}^2\sigma_{x\x{m}}^2\sigma_{y\x{f}}^2}{(\sigma^2+\sigma_{x\x{m}}^2)^2}\right) & \frac{\rho_\x{m}\sigma_{x\x{m}}\sigma_{y\x{m}}\sigma_{x\x{f}}\sigma_{y\x{f}}}{\sigma^2+\sigma_{x\x{m}}^2}& \sigma_{y\x{f}}^2\frac{\rho_\x{m}\sigma_{x\x{m}}\sigma_{y\x{m}}}{\sigma^2+\sigma_{x\x{m}}^2}
\\ & & \sigma_{x\x{f}}^2 & \rho_\x{f}\sigma_{x\x{f}}\sigma_{y\x{f}}
\\ & & & \sigma_{y\x{f}}^2
\end{array}\right). \label{covariance}
\end{align}
(We derive these expressions in the Supporting Information, Section S1.) Note that the distribution of the two traits among mating females is the same as among all adult females. This follows from our assumption that all females have equal reproductive success. Among mating males, however, the expected song becomes a weighted average of the average male song and average female preference. The expected male preference increases by an amount that depends on the correlation between male song and male preference and on whether the average female preference is greater than or less than the average male song.

\subsection*{Transmission of traits to offspring.  } 
\ The distribution of the traits within mating pairs allows us to find the distribution of the traits within the offspring of these pairs. To do so, we need to specify how offspring acquire each trait. We will consider threes ways that song can be acquired and four ways that preference can be acquired. Song can be acquired by 
\begin{enumerate}[A.]
\item a male learning the song of a randomly chosen adult male (oblique learning),
\item a bird of either sex inheriting the song gene from its parents, 
\item or a male learning the song of its father (paternal learning).
\end{enumerate}
When song is genetic, we will assume that a bird's song is the average of its parents' songs. 
%Since the distribution of mating pairs is normal, the distribution of the traits among offspring will also be normal. 
%When songs are learned, the songs among males in the offspring generation will be normally distributed with mean $$\mu_{x\x{m}}(t+1)=\frac{\sigma^2}{\sigma^2+\sigma_{x\x{m}}^2}\mu_{x\x{m}}+\frac{\sigma_{x\x{m}}^2}{\sigma^2+\sigma_{x\x{m}}^2}\mu_{y\x{f}},$$ by Equation \ref{mean}. 
The variance of the distribution of songs among male offspring depends on how the trait is acquired. For the three ways of acquiring song, using Equation \ref{covariance}, the variance of songs among males of the offspring generation obey the following recursion equations: 
\begin{align*}
\x{A.} \ &\sigma_{x\x{m}}^2(t+1)=\sigma_{x\x{m}}^2
\\ \x{B.} \ &\sigma_{x\x{m}}^2(t+1)=\frac{1}{4}\sigma_{x\x{m}}^2\left(\frac{\sigma^2(\sigma^2+\sigma_{x\x{m}}^2)+\sigma_{ x\x{m}}^2\sigma_{y\x{f}}^2}{(\sigma^2+\sigma_{x\x{m}}^2)^2}\right)+\frac{1}{2}\sigma_{x\x{m}}^2\frac{\rho_\x{f}\sigma_{x\x{f}}\sigma_{y\x{f}}}{\sigma^2+\sigma_{x\x{m}}^2}+\frac{1}{4}\sigma_{x\x{f}}^2
\\\x{C.} \ &\sigma_{x\x{m}}^2(t+1)=\sigma_{x\x{m}}^2\left(\frac{\sigma^2(\sigma^2+\sigma_{x\x{m}}^2)+\sigma_{ x\x{m}}^2\sigma_{y\x{f}}^2}{(\sigma^2+\sigma_{x\x{m}}^2)^2}\right).
\end{align*}
When males acquire their song by learning from a randomly chosen adult male, the distribution of songs does not change over time, so the variance of songs stays the same. 

Preference can be acquired by 
\begin{enumerate}[I.]
\item a female learning the preference of its mother (maternal learning),
\item a bird of either sex inheriting the preference gene from its parents,
\item a female imprinting on the song of its father and preferring that song (paternal imprinting),
\item or a female imprinting on a randomly chosen adult male and preferring that song (oblique imprinting).
\end{enumerate}
When preference is maternally learned, a female acquires the preference that her mother has. This could also occur if there is a W-linked preference gene that passes from mother to daughter without any contribution from the father. For simplicity, we will continue to refer to this type of acquisition as maternal learning, but the dynamics would be the same with a sex-linked preference gene. As with song, when preference is genetic, we will assume that a bird's preference is the average of its parents' preferences. 
Using Equation 2, we can also find the variance of preferences among females of the offspring generation for the four ways of acquiring preference:
\begin{align*}
\x{I.} \ & \sigma_{y\x{f}}^2(t+1) = \sigma_{y\x{f}}^2
\\\x{II.} \ & \sigma_{y\x{f}}^2(t+1) = \frac{1}{4}\sigma_{y\x{m}}^2\left(\frac{(\sigma^2+\sigma_{x\x{m}}^2)^2-\rho_\x{m}^2\sigma_{x\x{m}}^2(\sigma^2+\sigma_{x\x{m}}^2)+\rho_\x{m}^2\sigma_{x\x{m}}^2\sigma_{y\x{f}}^2}{(\sigma^2+\sigma_{x\x{m}}^2)^2}\right)+\frac{1}{2}\sigma_{y\x{f}}^2\frac{\rho_\x{m}\sigma_{x\x{m}}\sigma_{y\x{m}}}{\sigma^2+\sigma_{x\x{m}}^2}+\frac{1}{4}\sigma_{y\x{f}}^2
\\\x{III.} \ & \sigma_{y\x{f}}^2(t+1) = \sigma_{x\x{m}}^2\left(\frac{\sigma^2(\sigma^2+\sigma_{x\x{m}}^2)+\sigma_{ x\x{m}}^2\sigma_{y\x{f}}^2}{(\sigma^2+\sigma_{x\x{m}}^2)^2}\right)
\\\x{IV.} \  & \sigma_{y\x{f}}^2(t+1) = \sigma_{x\x{m}}^2.
\end{align*}

\subsection*{Dynamics of variance. }
%\ The variance of the distribution of songs among offspring does not depend on how preference is acquired. Likewise, the variance of the distribution of preferences among offspring does not depend on how song is acquired. However, the covariance between these traits in each sex, $C_\x{m}$ and $C_\x{f}$, depend on how both traits are acquired. 
%\ Note that there can only be a correlation between the traits if at least one of them is genetic, since otherwise each sex only has one of the two traits. 
\ There are twelve possible combinations of the ways each trait can be acquired, which we will number from $1$ to $12$: 
\newline
\begin{equation*}
\begin{tabular}{|l|c|c|c|}
\hline\ \ \ \ \ \ \ \ \ \ \ \ \ \ \ \ \ \ \ \ \ song:  & A. obliquely learned  & B. genetic & C.  paternally learned
\\\hline I. pref maternally learned & 1 & 2 & 3
\\\hline II. pref genetic & 4 & 5 & 6
\\\hline III. pref paternally imprinted & 7 & 8 & 9
\\\hline IV. pref obliquely imprinted & 10 & 11 & 12
\\\hline
\end{tabular}
\end{equation*}
\newline
The covariances between the traits in each sex, $C_\x{m}$ and $C_\x{f}$, depend on how both traits are acquired. Note that since a male only has a preference when preference is genetic, $C_\x{m}$ is equal to $0$ unless preference is genetic. Similarly, $C_\x{f}$ is equal to $0$ unless song is genetic.
In the Supporting Information, Section S2, we derive recursion equations for $\sigma_x^2$, $\sigma_y^2$, and $C$ for each sex for each of the mechanisms of acquisition. Here we show this derivation for mechanism 2, as an example. This is the case where song is genetic and a female acquires the preference of its mother. Because both sexes acquire song in the same way, we can drop the sex-specific subscripts on $\sigma_x^2$. Because only females have preferences, we can drop the sex-specific subscripts on $\sigma_y^2$ without losing track of which sex we are referring to. Because males only have songs, $C_\x{m}=0$. Using Equation \ref{covariance}, the covariance between the traits among females of the next generation will be 
\begin{align*}
C_\x{f}(t+1)&=\frac{1}{2}\sigma_{x}^2\frac{\sigma_{y}^2}{\sigma^2+\sigma_{x}^2}+\frac{1}{2}\rho_\x{f}\sigma_{x}\sigma_{y}.
\end{align*} 
Using the fact that $C_\x{f}=\rho_\x{f}\sigma_{x}\sigma_{y}$, we now have three recursion equations for how $\sigma_{x}^2$, $\sigma_{y}^2$, and $C_\x{f}$ change over time:
\begin{align*}
\sigma_x^2(t+1)&=\frac{\sigma_x^2}{4}\left(\frac{\sigma^2(\sigma^2+\sigma_{x}^2)+\sigma_{x}^2\sigma_{y}^2}{(\sigma^2+\sigma_{x}^2)^2}+\frac{2C_\x{f}}{\sigma^2+\sigma_x^2}+1\right)
\\ \sigma_y^2(t+1)&=\sigma_y^2
\\ C_\x{f}(t+1)&=\frac{1}{2}\sigma_{x}^2\frac{\sigma_{y}^2}{\sigma^2+\sigma_{x}^2}+\frac{1}{2}C_\x{f}.
\end{align*}
In the Supporting Information, Section S2, we show that there are two stable equilibria of these dynamics. The first occurs when $\sigma_x^{2\star}=C_\x{f}^\star=0$. The second stable equilibrium occurs when
\begin{align*}
\sigma_x^{2\star}&=\frac{3\sigma_y^2-5\sigma^2+\sqrt{9(\sigma_y^2)^2-30\sigma^2\sigma_y^2+(\sigma^2)^2}}{6}
\\ C_\x{f}^\star&=\frac{\sigma_x^{2\star}\sigma_y^2}{\sigma^2+\sigma_x^{2\star}}.
\end{align*}

\citet{Aoki:2001ly} studied a quantitative genetic model in which animals have a genetic trait and a female's preferred trait is the average of her parents' traits. Because both the sexually selected trait and the preference are affected by both parents, this is most similar to our mechanism 5, in which both traits are genetic. However, there are three main differences between their model and ours. First, they only consider one mechanism of acquisition, whereas we consider all twelve mechanisms shown in the table above. Second, although \citet{Aoki:2001ly} consider viability selection against males based on their songs, we made the simplifying assumption that neither males nor females experience viability selection based on either their songs or preferences in order to explore the effects of the twelve mechanisms. Third, \citet{Aoki:2001ly} focused on how the average male song evolves over time, whereas we focus on how the variance among male songs evolves over time.


\section*{Results}
\subsection*{Variance of song distribution at equilibrium. }
\ For each of the twelve mechanisms of acquisition, the recursion equations for $\sigma_x^2$, $\sigma_y^2$, and $C$ have stable equilibria (Table \ref{equilibrium}. 
First, we focus on whether the equilibrium value $\sigma_x^{2\star}$ is greater than or equal to $0$. If $\sigma_x^{2\star}>0$, multiple songs are present in the population, whereas if $\sigma_x^{2\star}=0$, only one song is present in the population and there is no diversity. As noted above, when song is obliquely learned (mechanisms $1$, $4$, $7$, and $10$) the variance of the distribution of songs does not change over time ($\sigma_x^{2\star}=\sigma_x^2(0)$). In other words, oblique learning of song permits the variance of the distribution of songs to persist undiminished across generations. Of the other eight mechanisms, when song is genetic or paternally learned, three have $\sigma_x^{2\star}>0$:

\begin{enumerate}
\item
In mechanism $2$ (song is genetic and preference is maternally learned), the system of recursion equations for $\sigma_x^2$ and $C$ is bistable. ($\sigma_y^2$ does not change over time.) There are two stable equilibria: one where $\sigma_x^{2\star}=0$ and another where $\sigma_x^{2\star}>0$, although the latter only exists when $\sigma_y^2\geq\frac{5+2\sqrt{6}}{3}\sigma^2$. If the initial values of $\sigma_x^2$ and $C$ are small, the dynamics will reach the first equilibrium, and if they are high enough, the dynamics will reach the second equilibrium. (A phase portrait of $\sigma_x^2$ and $C$ is shown in Figure \ref{mode2}.) 
\item In mechanism $3$ (song is paternally learned and preference is maternally learned), the recursion equation for $\sigma_x^2$ only has one stable equilibrium, $\sigma_x^{2\star}=\max\{\sigma_y^2-\sigma^2,0\}$. $\sigma_y^2$ does not change over time and $C$ is always equal to $0$.
\item In mechanism $5$ (song is genetic and preference is genetic), the system of recursion equations for $\sigma_x^2$, $\sigma_y^2$, and $C$ is bistable. There is a stable equilibria where $\sigma_x^{2\star}=\sigma_y^{2\star}=C=0$, which is reached if the initial values of $\sigma_x^2$, $\sigma_y^2$, and $C$ are small. If the initial values $\sigma_x^2$, $\sigma_y^2$, and $C$ are large enough, all three will increase indefinitely. (A phase portrait of $\sigma_x^2$ and $\sigma_y^2$ is shown in Figure \ref{mode5}.)
\end{enumerate}
Phase portraits for mechanisms $6$, $8$, $9$, $11$, and $12$ are shown in Figures \ref{mode6}-\ref{mode12}.
Changing how songs are acquired can either allow for or prevent the maintenance of diversity at equilibrium. When preference is maternally learned, for some parameters, diversity can be maintained when song is paternally learned but not when song is genetic (in Figure \ref{sigmax2_sigma2}A for some values of $\sigma^2$ a light blue line is above a dark red line). On the other hand, when preference is genetic, for some parameters, diversity can be maintained when song is genetic but not when song is paternally learned (in Figure \ref{sigmax2_sigma2}B for some values of $\sigma^2$ there is only a dark blue line and no red line). In sum, the effect of how song is acquired on the maintenance of diversity depends strongly on how preference is acquired.

%Similarly, when preference is maternally learned, song variation can persist ($\sigma_x^{2\star}>0$), regardless of how song is acquired.  When song is genetic, both maternal inheritance and genetic inheritance of preference permit persistence of song variation. When preference is genetic, both oblique learning and genetic inheritance of song permit persistence of song variation.  When song is paternally learned, only maternal inheritance of preference permits persistence of song variation. When preference is paternally imprinted, only oblique song learning permits persistence of song variation.  

Second, we compare the actual amount of variance, $\sigma_x^2$, when song is paternally learned to its value when song is genetic. We find that song variance is greater when song is paternally learned than when song is genetic (Figure \ref{sigmax2_sigma2}). In cases where the equilibrium variance of songs is greater than $0$, $\sigma_x^{2\star}$ is higher when song is paternally learned (compare the light red and blue curves in Figures \ref{sigmax2_sigma2}A). In cases where the equilibrium variance of songs is equal to $0$, the transient value of $\sigma_x^2$ after a finite number of generations is higher when song is paternally learned (compare the dark red and blue curves in Figures \ref{sigmax2_sigma2}C, E, and F). In these cases, it is additionally true that it takes more generations for the variance of the distribution of songs to approach $0$ when song is paternally learned than when song is genetic (compare the dark red and blue curves in Figure \ref{sigmax2_sigma2}B, D, F and G). 
 

\subsection*{Female choosiness. }
\ Both equilibrium and transient variance increase as $\sigma^2$ approaches $0$ (left column Figure \ref{sigmax2_sigma2}). (In the Supporting Information, Section S3, we show that the derivatives of $\sigma_x^{2\star}$ with respect to $\sigma^2$ are negative in those cases where $\sigma_x^{2\star}>0$.) In those cases when song diversity cannot be maintained, the number of generations it takes for the variance of the distribution of songs to become very small also increases as $\sigma^2$ approaches $0$ (right column of Figure \ref{sigmax2_sigma2}). In other words, as females become more and more choosy, the number of songs that persist in the population increases.  The exception to this rule is when preference is genetic and song is paternally learned (blue curves in Figure \ref{sigmax2_sigma2}C and D).  (We show equilibrium and transient values of $\sigma_x^{2}$  as a function of initial values $\sigma_x^2(0)$ and $\sigma_y^2(0)$ in Figures \ref{sigmax2_sigmax2}-\ref{sigmax2_sigmay2} and equilibrium and transient values of $\sigma_y^{2}$ as a function of $\sigma^2$, $\sigma_x^2(0)$, and $\sigma_y^2(0)$, in Figures \ref{sigmay2_sigma2}-\ref{sigmay2_sigmay2}.)
 

\subsection*{Multimodal distributions of songs at equilibrium. }
\ Until this point, we have assumed that both male song and female preference are initially normally distributed. With the addition of a Gaussian preference function, this ensures that both traits continue to be normally distributed from one generation to the next. Here we explore whether the simple mating rules captured in Equation \ref{model} can give rise to other types of distributions. Specifically, we use step functions (i.e.\ piecewise constant functions) for the initial distribution of each trait and analyze the resulting distributions. In these cases, it is not possible to write down an analytical expression for the distributions of traits in the offspring generation. Instead, it becomes necessary to find these numerically. The details about our numerical analysis are provided in the Supporting Information, Section S4. To make the comparison between a step function and a normal distribution as direct as possible, we construct a step function with the same mean and variance and such that the step function decreases with distance from the mean.  
  
When we use step functions for either the initial distribution of male songs or the initial distribution of female preferences, in mechanisms 3 and 5, the distribution of songs becomes multimodal (Figure \ref{mechs3and5}). If we were to observe a population with this distribution of songs, we would find males singing a few discrete song types and no males singing intermediate song types. The distribution of songs would appear to be discrete, even though both traits are technically continuous. In mechanism 2, when both traits are initially normally distributed, the dynamics of variance and covariance are bistable and can reach an equilibrium at which the variance of songs is greater than $0$. Using step functions for the trait distributions essentially moves the variance of songs into the basin of attraction of $0$. The distribution of songs collapses to a single point before peaks can form. Similarly, in the mechanisms where the variance of the distribution of songs goes to $0$, the distributions collapse to a single point before peaks can form, so multimodal distributions are not observed.  

The peaks in mechanism 3 can be explained as following. (Mechanism 5 follows similar principles.) Recall that the probability of a male with song $x$ mating with a female with preference $y$ is equal to the term $P_\x{f}(y)/Z_y$ multiplied by her preference $f_y(x)$ for him, which is maximized when $x=y$. A male therefore has greater mating success when $P_\x{f}(y)/Z_y$ is high for females with preferences close to his song. When the distribution of male songs follows a step function, $Z_y$ is low for some females (in Figure \ref{peak_example_song}C the black curve is sometimes below the red curve), i.e. it is difficult for some females to find males to whom they are attracted. For these females, $P_\x{f}(y)/Z_y$ is high (in Figure \ref{peak_example_song}D the black curve is sometimes above the red curve), and males whose songs are near these preferences increase in frequency (Figure \ref{peak_example_song}B). Similarly, when the distribution of female preferences follows a step function, there is an overabundance of females with preferences in a few discrete ranges (Figure \ref{peak_example_pref}A). For these females $P_\x{f}(y)/Z_y$ is high (Figure \ref{peak_example_pref}D) and males whose songs are near these regions increase in frequency over time (Figure \ref{peak_example_pref}B and D). 

The essence of this phenomenon is that the distribution of preferences has big shoulders, i.e. has negative excess kurtosis. The Pearson type VII family of distributions allows us to explicitly tune the kurtosis of the distribution of preferences. Using distributions from this family as the initial distribution of preferences in mechanism 3, we find that the equilibrium distribution of songs is multimodal only when the distribution of preferences has sufficiently negative excess kurtosis (Figure \ref{kurtosis}). If we allow for mutations in the males' songs, the peaks eventually disappear and the distribution becomes unimodal, but peaks can persist for thousands of generations (Figure \ref{mut_sensitivity}). A more detailed analysis of the effects of using a step function as the initial distribution of either trait is presented in Section S5. We also show there that using a step function for the females' preference function does not make the distribution of either trait multimodal.

\section*{Discussion}

It has been argued that whether bird songs are learned or genetically inherited determines the evolutionary dynamics of the trait (reviewed in \cite{Wilkins:2012ve,Lachlan:2004tg}). We compared the evolutionary dynamics of songs that are learned and genetically inherited by building a quantitative genetic model in which both song and preference are continuous traits that affect the likelihood of males and females mating and which can be passed to offspring in several different ways. There are several previous studies in which the authors compared two mechanisms for how songs and preferences are acquired. No previous studies to our knowledge have compared all twelve possible mechanisms that we consider here. Additionally, no such comparative studies have treated either song or preference as continuous variables. Our main finding is that the effect of song learning depends on how female preferences are acquired and on who song is learned from. The one overarching pattern we find is that the amount of variance in songs is higher when song is paternally learned than when song is genetic. We also find that the variation in the distribution of male songs increases as females become more choosy. Finally, we find two mechanisms of acquisition---when song is paternally learned and preference is maternally learned and when both traits are genetic---in which the distribution of songs can become multimodal.  

%This agrees with previous theoretical studies that have found that the likelihood of a population speciating strongly depends on how the female preferences are acquired \cite{Verzijden:2005vn,Kirkpatrick:1994vn}. 

\subsection*{Comparison to previous models of bird song. }
\ How multiple song types or dialects can persist within a single species of bird is poorly understood \cite{Weissing:2011hc,Planque:2014qf}. Several mechanisms have been proposed to explain this phenomenon. For example, if there are distinct subpopulations whose members migrate back and forth to only a limited degree, variety can be maintained \cite{Planque:2014qf,Yeh:2015bh,Ellers:2003zr}. If dispersal depends on mating success, subpopulations that sing different songs can be maintained \cite{Payne:1997fk}. Another mechanism that maintains diversity is viability selection against more-preferred songs \cite{Planque:2014qf,Yeh:2015bh}. Negative frequency-dependent natural selection can also generate variation \cite{Verzijden:2005vn}. Our results show that such a mechanism is not always required in order to maintain variation. In fact, we identify seven scenarios in which variation in the distribution of male songs can persist: when male songs are obliquely learned, when female preferences are maternally learned, and when both song and preference are genetic.

While there are multiple scenarios that allow for the persistence of variation, there is no simple rule that explains when diversity in songs can or cannot persist. This may explain why \citet{Verzijden:2012uq} in their review of mathematical models of sexual selection found no clear patterns for how song learning affects evolution. If we only consider a particular way in which preference is acquired, for example through genetic inheritance, and compare what happens when song is paternally learned and when song is genetic (mechanisms 5 and 6), it would be tempting to conclude that variation in songs can be maintained when song is genetic but not when song is learned. However, if we do the same comparison when song is maternally learned we would reach the opposite conclusion. Only by studying all twelve mechanisms of acquisition could we identify the interaction between how the two traits are acquired. 

Our results lead to predictions that we are interested in testing empirically. For example, we predict that there should be a single species-specific song in species where song is genetic and preference is paternally imprinted, while there should be multiple song types in species in which both traits are genetic. Generally, we predict that there should be a greater number of songs in species where song is paternally learned than in those where song is genetic.
 
While no previous studies have considered all twelve mechanisms, we can compare our results to pairwise comparisons made in previous studies. In those models, it was usually assumed that males can sing one of two songs and each female prefers one song over the other. Encouragingly, our results are generally similar to previous findings, even though we treat both traits as continuous variables. In their influential paper, \citet{Lachlan:2004tg} studied a model in which a single gene controls both female preferences and males' predisposition to sing different songs, which they can then acquire either genetically or by learning obliquely. The case where song is acquired genetically is similar to our mechanism 5. The case where song is obliquely learned is related to our mechanism 4, although there are substantial differences. They find that variance in both song and preference distributions disappear more slowly when song is obliquely learned (mechanism 4) compared to when song is genetic (mechanism 5). Because they only consider three possible songs, the variance in the distribution of songs cannot increase indefinitely, as we find in mechanism 5. Excluding this possibility, we find that variance in the distribution of songs persists when song is obliquely learned (mechanism 4) but not when song is genetic (mechanism 5), which is in rough agreement with their findings.   \citet{Yeh:2015bh} studied a model in which two diverging populations are coming back into contact. They considered a situation where females prefer to mate with males that have the same trait as they do, and compared situations where the trait is genetic or paternally learned. Although there are considerable differences between their model and ours, the two cases they focused on are most similar to our mechanisms 5 and 9. They found that divergence between the two populations is more difficult to maintain in their version of mechanism 5 than in their version of mechanism 9. Again excluding the case where variance increases indefinitely, we find that variance disappears more quickly in mechanism 5 than in mechanism 9, again showing rough agreement between our model and a quite different one. \citet{Verzijden:2005vn} studied two cases similar to our mechanisms 5 and 8: song is genetic and preference is either genetic or imprinted from the father's song. They found, as we did, that variance in the distribution of male songs can be maintained in mechanism 5, but not in mechanism 8. \citet{Kirkpatrick:1994vn} studied songs that are male sex-linked and preferences that are acquired by observing randomly chosen successfully-mating males, which is most similar to our mechanism 9. They found that variation in male songs could not be maintained, which agrees with our results. Overall, the results from these several models, in which mating traits are treated discretely, and the results from our model, in which mating traits are treated continuously, seem to agree, which is encouraging evidence that these results are quite robust. By studying a model of continuously distributed traits, we have expanded our understanding of how the mechanisms of acquisition themselves, rather than the types of traits under consideration, determine the evolutionary dynamics of song.

\subsection*{Evolutionary branching. }
\ We find that, in general, there is more variance in the distribution of songs when females are choosier. This is somewhat surprising, as it might seem that choosier females would exert stronger selection pressure and would reduce the number of songs that can persist in the population. However, we find the opposite.
%``Classical female choice models already demonstrated that a single runaway process will occur only if the initial level of choosiness exceeds a certain threshold value \cite{Kirkpatrick:1982kl}" \cite{van-Doorn:2004tg} %Kirkpatrick reference needs to be checked  
Our result is closely related to the results of previous studies on evolutionary branching. Evolutionary branching refers to a situation where a monomorphic population divides into two subpopulations that have different phenotypes and such that breeding only occurs between members of the same subpopulation \cite{Doebeli:2000oq,Doorn:2000nx,Weissing:2011hc}. How specialized animals are can determine whether evolutionary branching occurs. For example, if the members of a population compete for resources and vary in a trait that determines the type of resources an animal can consume (e.g. a bird's bill determines what size seed it can eat), then two subpopulations (e.g. large- and small-billed) can co-exist only if the animals are specialized in their consumption of resources, and otherwise a single optimally-adapted type will reach fixation \cite{Doebeli:2000oq,Doorn:2000nx,Weissing:2011hc}. Our results contribute to a growing understanding that this trend is true for sexual selection as well as for natural selection. In this case, males are competing for females, not resources: if females are choosy, males ``specialize" in traits that are preferred by a small subset of females, but if females are promiscuous, males can mate with many types of females and are ``generalists." In this way, we find that when males sing ``specialized" songs, more song types can persist in the population. Similarly, \citet{Doorn:2000nx} analyzed the adaptive dynamics of a mating trait and preference trait when each is haploid genetic (similar to our mechanism 5) and found that evolutionary branching only occurs when females are sufficiently choosy. Empirically, \citet{Van-Doorn:2001fv} found a greater diversity of sperm proteins under sexual selection when they were under greater constraints to match female egg proteins. Although general principles of evolution occurring through both natural and sexual selection have been hard to come by \cite{Kirkpatrick:2002fu}, one commonality seems to be that animals that specialize, either in their consumption of  resources or in their appeal to mating partners, can maintain diversity more easily.


Surprisingly, even when both songs and preferences are continuously distributed, we find that the population can reach an equilibrium in which songs are multimodally distributed. In other words, a few discrete song types are present in the population, while intermediate songs are not. Many species of birds produce discrete song types (for example \cite{Gibbs:1990fk,Devoogd:1993uq}). One explanation might be that they are only capable of producing a discrete set of songs. Another might be that females only like a discrete set of songs. Our results show that discrete song types can be favored even if males are capable of producing a continuous range of songs and there are females that prefer every song on the spectrum.

We found multimodal equilibria by using representatives of two very different probability distributions as the initial distributions of songs and preferences (step functions and Pearson type VII distributions). We therefore expect that this phenomenon should be quite general. Indeed, slight deviations from normal distributions, as would be expected in the evolution of finite populations, may be sufficient to lead to a multimodal distribution of songs. The multimodal distributions of songs that we find in our model do not qualify as evolutionary branching because it is still possible for any two members of the population to mate with each other. However, the absence of intermediate males may make it easier for evolutionary branching to occur. This interesting question is left for future work. 
 
 
\subsection*{Speciation. } 
\ In this paper, we study how the amount of variation in male songs depends on how the songs and preferences are acquired and on how promiscuous females are. Our results are relevant to understanding the standing variation within any given species of birds, which we can test against empirical patterns as described above. Ultimately, we are interested in how these patterns affect the rates at which different groups of birds have speciated. Greater variance in male songs can both accelerate and decelerate speciation. On the one hand, two subpopulations can only diverge with respect to their songs if they sing different songs, so some amount of variance is required for speciation \cite{Mead:2004uq,Slabbekoorn:2002kl}. If a population produces multiple song types and they become associated with particular habitat types, song divergence can be accompanied by ecological divergence, which can result in speciation \cite{Slabbekoorn:2002kl,Doorn:2000nx}.
On the other hand, if two subpopulations are beginning to diverge and the range of songs produced by males in each subpopulation overlap, it will be more difficult for speciation to occur via sexual selection \cite{Verzijden:2012uq,Olofsson:2011kx,Kirkpatrick:2002fu,Irwin:1999fk}. Two important factors determining whether speciation occurs are the amount of ecological diversification that accompanies diversification in mating traits and the correlation that does (or does not) arise between female preferences and male traits \cite{Doorn:2000nx,Doebeli:2000oq,Verzijden:2012uq}.  
%Frequency-dependent selection seems to facilitate evolutionary branching in both female preferences and male traits \cite{van-Doorn:2004tg,Weissing:2011hc}. 

Based on our analysis of the case where both traits are genetic, we tentatively suggest that speciation may occur most easily under this mechanism. It is the only situation in which variation in both traits can continue to increase indefinitely. This occurs because a perfect correlation between female preference and male song arises. Unlike a normal case of ``runaway" selection, when the mean value of a trait under selection becomes more and more exaggerated over time because of a correlation between mating preferences and the traits under sexual selection \cite{Lande:1981fk,Doorn:2000nx,Aoki:2001ly,Mead:2004uq}, here it is the variance of the trait that increases over time. Additionally, this is one of two cases we find in which the distribution of songs can become multimodal. It seems likely that a population whose males sing two or three discrete songs, rather than a continuous spectrum of songs, could speciate rapidly if subpopulations diverge based on which of the discrete song types their males sings. This agrees with previous findings that, among situations where song is genetic, the case where preference is also genetic is fundamentally different in how readily it leads to speciation \cite{Verzijden:2005vn,Gilman:2015fk}. At first glance, our suggestion seems to contradict the fact that the order of birds in which males usually learn their songs, the Passeriformes, has many more species than any other order \cite{Wilkins:2012ve,Lachlan:2004tg} and high  speciation rates \cite{Jetz:2013fv}. To understand this contradiction, we need to do further modeling to make more rigorous predictions about the effects of song acquisition on speciation. Specifically, we plan to extend this model to include two subpopulations that are beginning to diverge in order to identify the circumstances under which they are or are not likely to completely diverge and become two separate species. Additionally, rather than a simple comparison of speciation rates between species in which males learn their songs and those in which song is genetic, what is needed is a careful analysis of the speciation rates based on how both song is acquired and preference is acquired. Female preference is critical to the outcome of our models, but it is not often included in empirical studies of diversification.

\subsection*{Future directions and conclusion. }
\ We constructed a mathematical model of the evolutionary dynamics of bird song driven by female mate choice, so that our results could be more easily interpreted and generalized across many species of birds. In addition to considering multiple interbreeding populations, we are interested in extending the model in several ways. First, we would like to consider songs that vary in several dimensions. For example, different males can sing different pitches and can repeat their songs at different rates. How the dimensionality of a trait affects its evolutionary dynamics is an open question in evolutionary theory and this would be an interesting model system in which to address this question. Second, we assumed that each male can only sing a single song. However, in many species of birds, any given individual can produce multiple songs. How the number of songs any individual can sing evolves is an interesting question. 
%Next, we only considered two types of female preference function, a Gaussian and a step function. Encouragingly, we found that this did not have a strong effect on our results.  However, in the rare cases where female preference functions for male songs have been measured (for example in bushcrickets \cite{Ritchie:1996ys}), the preference function does not appear to be Gaussian. We are interested in more thoroughly testing the effects of this factor. 
Third, in our model, when either song or preference is genetic, an offspring acquires the average of its parents' traits. Other genetic architectures may result in different behaviors.
 

There are many interesting biological traits that have learned components. In addition to bird song, other examples include migratory behavior \cite{Mueller:2013bh}, the structures bowerbirds use to attract mates \cite{Madden:2008ij}, and nest-site choice \cite{Seppanen:2007zr}. As with bird song, divergence in any of these traits can lead to speciation, so the question of how learning affects speciation is quite general. Humans are unique in the extent to which our behavior is shaped by learning rather than by genes \cite{Laland:2010fu}, so understanding the effects of learning on evolution is critical to understanding our own evolutionary history. There are striking similarities between biological evolution and cultural evolution, and the same mathematical models have been used to study both with great success \cite{Mesoudi:2006fk}. However, as we have found, there can be dramatic differences in the dynamics of traits that are genetically inherited and those that are learned. Comparing and contrasting these dynamics improves our understanding of how cultural evolution relates to biological evolution and when it leads to a fundamentally different outcome.


\bibliographystyle{unsrtnat}
\bibliography{song_learning_evolution}

\newpage

\begin{table}[tp]
\caption{\label{variables} Table of variables used in the text. For the sake of brevity, we only give the male version of any sex-specific variable. For example, while we use both $\sigma_{x\x{m}}^2$ and $\sigma_{x\x{f}}^2$ in the text, we only include $\sigma_{x\x{m}}^2$ in the table.}
\vspace{5pt}
\begin{tabular}{lllll}
Variable & Interpretation
\\\hline $C_\x{m}=\rho_\x{m}\sigma_{x\x{m}}\sigma_{y\x{m}}$ & covariance of traits among adult males
\\ $C^\star$ & equilibrium covariance of traits
\\ $\mu$ & mutation rate 
\\$\mu_\x{m}=(\mu_{x\x{m}},\mu_{y\x{m}})^T$ & vector of average values of each trait among adult males 
\\$m,n$ & widths of steps in step function 
\\ $\rho_\x{m}=\frac{C_\x{m}}{\sigma_{x\x{m}}\sigma_{y\x{m}}}$ & correlation between traits among adult males
\\ $\rho^\star$ & equilibrium correlation between traits
\\$\Sigma_\x{m}$ & covariance matrix of traits among adult males
\\$\Sigma_\text{mate}$ & covariance matrix of traits among mating adults
\\$\sigma_{x\x{m}}^2$ & variance of songs among adult males
\\$\sigma_{x}^{2\star}$ & equilibrium variance of songs
\\$\sigma_{y\x{m}}^2$ & variance of preferences among adult males
\\$\sigma_y^{2\star}$ & equilibrium variance of preferences
\\$\sigma^2$ & variance of preference function
\\$u=(x_\x{m},y_\x{m},x_\x{f},y_\x{f})^T$ & vector of both traits in both sexes
\\$v=(x,y)^T$ & vector of traits
\\$x$ & song
\\$y$ & preference
\\$Z_y=\int_{\R^2}P_\x{m}(v_\x{m})f_{y}(x_\x{m})dv_\x{m}$ & integral of males a female with preference $y$ finds attractive
\\ $\delta$ & width of steps of the partition $S$ for numerical analyses
\\ $M$ & half-width of interval of traits for numerical analyses
\\ $S$ & partition of discrete set of traits for numerical analyses
\end{tabular}
\end{table}

\begin{landscape}
\begin{table}
\caption{\label{equilibrium}Here we show the stable equilibria of the twelve possible mechanisms of acquisition. Stable equilibria are indicated with a ${}^\star$ and initial values are indicated with $(0)$. For example $\sigma_x^{2\star}$ is a stable equilibrium of $\sigma_x^2$, and $\sigma_x^2(0)$ is the initial value of $\sigma_x^2$. The rows and columns are ordered so that the equilibrium variance of the distribution of songs $\sigma_x^{2\star}=0$ in cells that are lower and to the right.  We write ND when $\rho$ is not defined, because either $\sigma_x^{2\star}$ or $\sigma_y^{2\star}$ or both are equal to $0$. There are two mechanisms in which the system of recursion equations for $\sigma_x^2$ and $C$ are bistable: when song is genetic and preference is maternally learned and when both song and preference are genetic. Both stable equilibria are included in the table. The second equilibrium when song is genetic and preference is maternally learned only exists when $\sigma_y^2\geq\frac{5+2\sqrt{6}}{3}\sigma^2$.  }
%\vspace{5pt}
%\hspace{-50pt}
\begin{tabular}{|l|l|l|l|}
\hline \ \ \ \ \ \ \ \ \ \ \ \ \ \ \ \ \ \ \ \ \ song: & A.  obliquely learned  & B. genetic & C. paternally learned
\\\hline 
pref: &&&
\\I. maternally learned  & $\sigma_x^{2\star}=\sigma_x^2(0)$ & $\sigma_x^{2\star}=0, \ \frac{3\sigma_y^2-5\sigma^2+\sqrt{9(\sigma_y^2)^2-30\sigma^2\sigma_y^2+(\sigma^2)^2}}{6}$ & $\sigma_x^{2\star}=\max\{\sigma_y^2-\sigma^2,0\}$  
\\ 	& 	$\sigma_y^{2\star}=\sigma_y^2(0)$ 	& $\sigma_y^{2\star}=\sigma_y^2(0)$ 		  & $\sigma_y^{2\star}=\sigma_y^2(0)$   
\\ & $ C^\star=0$ &   $ C^\star=\frac{\sigma_x^{2\star}\sigma_y^{2\star}}{\sigma^2+\sigma_x^{2\star}}$  & $ C^\star=0$
\\ & $\rho^\star=0$ & $\rho^\star=\text{ND},\frac{\sigma_x^{\star}\sigma_y^{\star}}{\sigma^2+\sigma_x^{2\star}}$ & $\rho^\star=0$ 
\\\hline II. genetic &  $\sigma_x^{2\star}=\sigma_x^2(0)$  & $\sigma_x^{2\star}=0,\ \infty$  & $\sigma_x^{2\star}=0$                      
\\  		&  $\sigma_y^{2\star}=0$	& $\sigma_y^{2\star}= 0 , \ \infty$ 	  & $\sigma_y^{2\star}=0$  
\\ & $ C^\star=0$   & $ C^\star=0, \ \infty$        & $ C^\star=0$ 
\\ & $\rho^\star=\text{ND}$ & $\rho^\star=\text{ND},1$ & $\rho^\star=\text{ND}$         
\\\hline III. paternally imprinted & $\sigma_x^{2\star}=\sigma_x^2(0)$ & $\sigma_x^{2\star}=0$  & $\sigma_x^{2\star}=0$                       
\\  			& $\sigma_y^{2\star}=\frac{\sigma_x^2(\sigma^2+\sigma_x^2)}{2\sigma_x^2+\sigma^2}$	  & $\sigma_y^{2\star}=0$  & $\sigma_y^{2\star}=0$                       
\\ & $ C^\star=0$ & $ C^\star=0$ & $ C^\star=0$
\\ & $\rho^\star=0$ & $\rho^\star=\text{ND}$ & $\rho^\star=\text{ND}$
\\ \hline IV. obliquely imprinted & $\sigma_x^{2\star}=\sigma_x^2(0)$ & $\sigma_x^{2\star}=0$ & $\sigma_x^{2\star}=0$
\\ & $\sigma_y^{2\star}=\sigma_x^2(0)$ & $\sigma_y^{2\star}=0$ & $\sigma_y^{2\star}=0$
\\ & $C^\star=0$ & $C^\star=0$ & $C^\star=0$
\\ & $\rho^\star=0$ & $\rho^\star=ND$ & $\rho^\star=ND$
\\\hline
\end{tabular}
\end{table}
\end{landscape}


\begin{figure}
\includegraphics[width=6.5in]{sigmax2_by_female_mode.pdf}
\caption{\label{sigmax2_sigma2}  Song variance tends to be higher when song is paternally learned than when it is genetic. In the left column, we show either equilibrium song variance, $\sigma_x^{2\star}$, or transient song variance, $\sigma_x^2(25)$, as a function of the variance of the preference function, $\sigma^2$. In the right column, we show the number of generations it takes before $\sigma_x^2$ is less than $0.05$, for those cases where $\sigma_x^{2\star}=0$. In the top row, A and B, preference is maternally learned. In the second row, C and D, preference is genetic. In the third row, E and F, preference is paternally imprinted. In the fourth row, G and H, preference is obliquely imprinted. Red indicates song is genetic; blue indicates song is paternally learned. Mechanisms in which song is obliquely learned are not presented here. We use lighter colors for parameters where $\sigma_x^{2\star}>0$ and darker colors for parameters where $\sigma_x^{2\star}=0$. In the second row, C and D, the dark red curves (meaning both song and preference are genetic) start at $\sigma^2\approx 1.3$ because, at smaller values of $\sigma^2$, $\sigma_x^{2\star}=\infty$. In this figure both traits are normally distributed and the preference function is Gaussian.  Parameters: $\rho(0)=0.6$, $\sigma_x^2(0)=1$, $\sigma_y^2(0)=4.1$. }
\end{figure}

\begin{figure}
\includegraphics[width=6.8in]{effect_of_step_trait_distribution.pdf}
\caption{\label{mechs3and5} The distribution of male songs can reach multimodal equilibria. In each panel we show the frequency of the two traits, male song ($x$) in red and female preference ($y$) in blue. The panels in the first column show the initial distributions of the two traits: in A the distribution of male songs follows a step function and the distribution of female preferences is normal; in D the distribution of male songs is normal and the distribution of female preferences follows a step function. B and C show the distributions of songs and preferences that follow from the initial conditions in A; E and F show the distributions of songs and preferences that follow from the initial conditions in D. The second column shows the equilibrium distributions of songs and preferences in mechanism 3, where song is paternally learned and preference is maternally learned. The equilibrium distributions are found by iterating the mating process for $10000$ generations. The third column shows the distribution of songs and preferences in mechanism 5, where both song and preference are genetic, after $10$ generations. Over time, the variance of these distributions increases and the peaks continue to grow farther and farther apart. In this figure the preference function is Gaussian. Parameters: $\rho(0)=0.9$, $\sigma^2=0.5$, $\sigma_x^2(0)=1$, $\sigma_y^2(0)=2$, $\delta=0.23$, $M=24$, $m=0.71$,  $n=2.35$.}
\end{figure}

\begin{figure}
\includegraphics[width=6.8in]{peak_example_step_song_dist.pdf}
\caption{\label{peak_example_song}  The equilibrium distribution of songs in mechanism 3 is multimodal when the initial distribution of male songs is a step function. In A we show the initial distribution of songs. The red curve shows a normal distribution and the black curve shows a step function. In B-D, red curves indicates the conditions that follow from the normal distribution and black curves indicates the conditions that follow from the step function. In B we show the equilibrium distribution of songs, which is found by iterating the mating process for $10000$ generations. In C we show the fraction of the initial distribution of males that females in the initial generation find attractive, $Z_y$, as a function of female preference, $y$. In D we show, for each preference $y$, the frequency of females in the initial generation with that preference, $P_\x{f}(y)$, divided by $Z_y$. The y-axis in D is on a logarithmic scale. The peaks in this curve lead to peaks in the distribution of songs.  In this figure the initial distribution of preferences is normal and the preference function is Gaussian. The distribution in B shows the Parameters: $\rho(0)=0.9$, $\sigma^2=0.5$,  $\sigma_x^2(0)=1$, $\sigma_y^2(0)=2$,  $\delta=0.2$, $M=14$, $m=0.8,$ $n=2.6$. 
}
\end{figure}

\begin{figure}
\includegraphics[width=6.5in]{peak_example_step_pref_dist.pdf}
\caption{\label{peak_example_pref}  The equilibrium distribution of songs in mechanism 3 is multimodal when the initial distribution of female preferences is a step function. In A we show the initial distribution of preferences. The red curve shows a normal distribution and the black curve shows a step function. In B-D, red curves indicates the conditions that follow from the normal distribution and black curves indicates the conditions that follow from the step function. In B we show the equilibrium distribution of songs, which is found by iterating the mating process for $10000$ generations. In C we show the fraction of the initial generation of males that females find attractive, $Z_y$, as a function of female preference, $y$. There is only one curve because the initial of distribution of males is normal for both initial distributions of preferences. In D we show, for each preference $y$, the frequency of females in the initial generation with that preference, $P_\x{f}(y)$, divided by $Z_y$. The y-axis in D is on a logarithmic scale. The peaks in this curve lead to peaks in the distribution of songs. In this figure the initial distribution of songs is normal and the preference function is Gaussian. Parameters: $\rho(0)=0.9$, $\sigma^2=0.5$, $\sigma_x^2(0)=1$, $\sigma_y^2(0)=2$, $\delta=0.2$, $M=14$, $m=0.8,$ $n=2.6$. 
}
\end{figure}



\end{document}
