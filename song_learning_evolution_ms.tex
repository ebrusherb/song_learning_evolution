\documentclass{article}

\usepackage[textwidth=7in,textheight=10in]{geometry}
\usepackage{graphicx}
\usepackage{enumerate}
\usepackage{/Users/eleanorbrush/Documents/custom2}
\usepackage{wasysym}
\usepackage{color}
\usepackage[numbers,sort&compress]{natbib}
\newcommand{\ra}[1]{\renewcommand{\arraystretch}{#1}}
\newcommand{\x}[1]{\text{#1}}
\newcommand{\Cov}{\text{Cov}}
\usepackage{lscape} 
\newcommand\numberthis{\addtocounter{equation}{1}\tag{\theequation}}

\usepackage{tocloft}% http://ctan.org/pkg/tocloft
\setlength{\cftsecnumwidth}{2em}% Set length of number width in ToC for \subsection
\cftsetindents{subsection}{3em}{2em}
\setcounter{tocdepth}{2}% Allow only \chapter in ToC

\begin{document}
%\tableofcontents

\section*{Introduction}
We find that 
\begin{enumerate}
\item There are more cases in which variance in the distribution of songs can be maintained when song is genetic as opposed to paternally learned.
\item The actual level of variance in the distribution of songs is higher in those cases when song is paternally learned as opposed to genetic.
\item There is the most variance in the distribution of songs when females are highly choosy when deciding with whom to mate.
\item The basic rule dictating how mating pairs form allows for both a normal distribution of songs at equilibrium and a multimodal distribution of songs at equilibrium.
\end{enumerate}

\section*{Model}

In our model, we consider a single population of birds. Each male bird has a song it uses to attract female birds. Each female has a preference for a particular song. All females get to mate and have equal mating success. On the other hand, males can mate with multiple females and those with more attractive songs will have higher mating success. We will consider cases where either or both of these traits are genetic. If the song is genetic, then both males and females carry a gene for song and females may in fact sing, but a bird's song only affects the mating success of males. If the preference is genetic, then both males and females carry a gene for preference, but a male bird's preference gene has no effect on who it mates with. After the adults mate and reproduce, they die and their offspring become next year's adults. In other words, we assume there are non-overlapping generations. We will consider a number of ways that offspring of these mating pairs can acquire both songs and preferences. Depending on how these traits are acquired, the distributions of songs and preferences in the population will change over time. In particular, the diversity of songs among offspring can either increase or decrease over time.

Mathematically, we will assume every bird has a song $x$ and preference $y$. (Table \ref{variables} defines all the variables used in the text.) We assume that initially both traits are normally distributed among the adults of the population. 
If $v=(x,y)^T$ is the two-dimensional vector giving the two traits, the probability distribution of these traits among all adult males is 
\begin{align*}
P_\x{m}(v)&=\frac{1}{2\pi\sqrt{|\Sigma_\x{m}|}}\exp\left(-\frac{1}{2}(v-\mu_\x{m})^T\Sigma_\x{m}^{-1}(v-\mu_\x{m})\right),
\end{align*} where $\mu_\x{m}=(\mu_{x\x{m}},\mu_{y\x{m}})^T$ gives the expected values of songs and preferences among adult males and 
\begin{align*}
\Sigma_{\x{m}}=\left(\begin{array}{cc}\sigma_{x\x{m}}^2 & \rho_\x{m}\sigma_{x\x{m}}\sigma_{y\x{m}} \\ \rho_\x{m}\sigma_{x\x{m}}\sigma_{y\x{m}} & \sigma_{y\x{m}}^2 \end{array}\right)
\end{align*}
is the covariance matrix for the traits in adult males. Note that the parameter $\rho_\x{m}$ is the correlation between songs and preferences among adult males. We will also use $C_\x{m}=\rho_\x{m}\sigma_{x\x{m}}\sigma_{y\x{m}}$ to denote the covariance between the traits in adult males. Similarly, the probability distribution of these traits among all adult females is 
\begin{align*}
P_\x{f}(v)&=\frac{1}{2\pi\sqrt{|\Sigma_\x{f}|}}\exp\left(-\frac{1}{2}(v-\mu_\x{f})^T\Sigma_\x{f}^{-1}(v-\mu_\x{f})\right), 
\end{align*}
where $\mu_\x{f}$ is the expected values of songs and preferences among adult females,  $\Sigma_\x{f}$ is the covariance matrix of these traits among adult females, $\rho_\x{f}$ is the correlation between the traits among adult females, and $C_\x{f}$ is the covariance between the trait in adult females. For most of the analyses that follow, we assume that each female uses a Gaussian preference function, centered at her preference $y$ with a variance $\sigma^2$:
\begin{align*}
f_y(x)&=\frac{1}{\sqrt{2\pi\sigma^2}}\exp\left(\frac{(x-y)^2}{2\sigma^2}\right).
\end{align*}
The variance $\sigma^2$ can be thought of a promiscuity---the larger $\sigma^2$ is, the more males a female is willing to mate with--- or as being inversely related to choosiness---the larger $\sigma^2$ is, the less choosy a female is. 

The probability of a male with traits $v_\x{m}$ and a female with traits $v_\x{f}$ mating is proportional to the product of the probabilities of finding such a male and such a female, with an additional factor describing the likelihood of such a female mating with such a male:
\begin{equation} \label{model}
P_\x{mate}(v_\x{m},v_\x{f})=\frac{P_\x{f}(v_\x{f})P_\x{m}(v_\x{m})f_{y_\x{f}}(x_\x{m})}{Z_{y_\x{f}}},
\end{equation}
where $Z_{y_\x{f}}=\int_{\R^2} P_\x{m}(v_\x{m})f_{y_\x{f}}(x_\x{m})dv_\x{m}$ is a normalizing factor that describes the total number of males a female with preference $y_\x{f}$ finds attractive.

In the Supporting Information, we derive the probability that a male with traits $v_\x{m}=(x_\x{m},y_\x{m})^T$ and a female with traits $v_\x{f}=(x_\x{f},y_\x{f})^T$ will mate. Since we use normal distributions for the initial conditions of both traits and for the preference function, the probability of a particular pair mating can also be described with a normal distribution. Specifically, if $u=(x_\x{m},y_\x{m},x_\x{f},y_\x{f})^T$ the distribution $P_\text{mate}(u)$ is a multivariate normal with expectation 
\begin{align}
\mu_\text{mate}&=\left(\begin{array}{cc} \frac{\sigma^2}{\sigma^2+\sigma_{x\x{m}}^2}\mu_{x\x{m}}+\frac{\sigma_{x\x{m}}^2}{\sigma^2+\sigma_{x\x{m}}^2}\mu_{y\x{f}} 
\\ \mu_{y\x{m}}+\frac{\rho_\x{m}\sigma_{x\x{m}}\sigma_{y\x{m}}}{\sigma^2+\sigma_{x\x{m}}^2}(\mu_{y\x{f}}-\mu_{x\x{m}})
\\ \mu_{x\x{f}}
\\ \mu_{y\x{f}}
 \end{array}\right) \label{mean}
\end{align}
and covariance $\Sigma_\text{mate}$:
\begin{align}
\left(\begin{array}{cccc}\sigma_{x\x{m}}^2\left(\frac{\sigma^2(\sigma^2+\sigma_{x\x{m}}^2)+\sigma_{x\x{m}}^2\sigma_{y\x{f}}^2}{(\sigma^2+\sigma_{x\x{m}}^2)^2}\right) & \frac{\rho_\x{m}\sigma_{x\x{m}}\sigma_{y\x{m}}(\sigma^2(\sigma^2+\sigma_{x\x{m}}^2)+\sigma_{x\x{m}}^2\sigma_{y\x{f}}^2)}{(\sigma^2+\sigma_{x\x{m}}^2)^2} & \sigma_{x\x{m}}^2\frac{\rho_\x{f}\sigma_{x\x{f}}\sigma_{y\x{f}}}{\sigma^2+\sigma_{x\x{m}}^2} &  \sigma_{x\x{m}}^2\frac{\sigma_{y\x{f}}^2}{\sigma^2+\sigma_{x\x{m}}^2}
\\ & \sigma_{y\x{m}}^2\left(\frac{(\sigma^2+\sigma_{x\x{m}}^2)^2-\rho_\x{m}^2\sigma_{x\x{m}}^2(\sigma^2+\sigma_{x\x{m}}^2)+\rho_\x{m}^2\sigma_{x\x{m}}^2\sigma_{y\x{f}}^2}{(\sigma^2+\sigma_{x\x{m}}^2)^2}\right) & \frac{\rho_\x{m}\sigma_{x\x{m}}\sigma_{y\x{m}}\sigma_{x\x{f}}\sigma_{y\x{f}}}{\sigma^2+\sigma_{x\x{m}}^2}& \sigma_{y\x{f}}^2\frac{\rho_\x{m}\sigma_{x\x{m}}\sigma_{y\x{m}}}{\sigma^2+\sigma_{x\x{m}}^2}
\\ & & \sigma_{x\x{f}}^2 & \rho_\x{f}\sigma_{x\x{f}}\sigma_{y\x{f}}
\\ & & & \sigma_{y\x{f}}^2
\end{array}\right). \label{covariance}
\end{align}
Note that both the expected values and the covariance structure of the traits among  mating females is the same as among all adult females. This is because of our assumption that all females have equal reproductive success. Among mating males, however, the expected song becomes a weighted average of the average male song and average female preference. The expected male preference increases by an amount that depends on the correlation between male song and male preference and on whether the average female preference is greater than or less than the average male song.

This distribution of the traits within mating pairs allows us to find the distribution of the traits within the offspring of these pairs. To do so, we need to specify how the offspring acquire each trait. We will consider three ways each trait can be acquired. The song can be acquired by 
\begin{enumerate}
\item a male learning the song of a randomly chosen adult (``obliquely" as opposed to vertically),
\item a bird of either gender inheriting the song gene from its parents, 
\item or a male learning the song of its father.
\end{enumerate}
When song is genetic, we will assume that a bird's song is the average of its parents' songs. Since the distribution of mating pairs is normal, the distribution of the traits among offspring will also be normal. For example, using Equations \ref{mean} and  \ref{covariance}, if a male learns the song of its father, the songs among males in the offspring generation will be normally distributed with mean $$\mu_{x\x{m}}(t+1)=\frac{\sigma^2}{\sigma^2+\sigma_{x\x{m}}^2}\mu_{x\x{m}}+\frac{\sigma_{x\x{m}}^2}{\sigma^2+\sigma_{x\x{m}}^2}\mu_{y\x{f}}$$ and variance $$\sigma_{x\x{m}}^2(t+1)=\sigma_{x\x{m}}^2\left(\frac{\sigma^2(\sigma^2+\sigma_{x\x{m}}^2)+\sigma_{ x\x{m}}^2\sigma_{y\x{f}}^2}{(\sigma^2+\sigma_{x\x{m}}^2)^2}\right).$$
We focus on how the variance of the two traits change over time. For the three ways of acquiring song, the variance of songs among males of the offspring generation obey the following recursion equations: 
\begin{align*}
1. \ &\sigma_{x\x{m}}^2(t+1)=\sigma_{x\x{m}}^2
\\ 2. \ &\sigma_{x\x{m}}^2(t+1)=\frac{1}{4}\sigma_{x\x{m}}^2\left(\frac{\sigma^2(\sigma^2+\sigma_{x\x{m}}^2)+\sigma_{ x\x{m}}^2\sigma_{y\x{f}}^2}{(\sigma^2+\sigma_{x\x{m}}^2)^2}\right)+\frac{1}{2}\sigma_{x\x{m}}^2\frac{\rho_\x{f}\sigma_{x\x{f}}\sigma_{y\x{f}}}{\sigma^2+\sigma_{x\x{m}}^2}+\frac{1}{4}\sigma_{x\x{f}}^2
\\3. \ &\sigma_{x\x{m}}^2(t+1)=\sigma_{x\x{m}}^2\left(\frac{\sigma^2(\sigma^2+\sigma_{x\x{m}}^2)+\sigma_{ x\x{m}}^2\sigma_{y\x{f}}^2}{(\sigma^2+\sigma_{x\x{m}}^2)^2}\right).
\end{align*}
When males acquire their song by learning from a randomly chosen adult male, the distribution of songs does not change over time, so the variance of songs stays the same. 

The preference can be acquired by 
\begin{enumerate}
\item a female learning the preference of its mother,
\item a bird of either gender inheriting the preference gene from its parents,
\item or a female learning the song of its father and preferring that song.
\end{enumerate}
As with song, when preference is genetic, we will assume that a bird's preference is the average of its parents' preferences. Using Equation 2, we can also find the variance of preferences among females of the offspring generation for the three ways of acquiring preference:
\begin{align*}
1. \ & \sigma_{y\x{f}}^2(t+1) = \sigma_{y\x{f}}^2
\\2. \ & \sigma_{y\x{f}}^2(t+1) = \frac{1}{4}\sigma_{y\x{m}}^2\left(\frac{(\sigma^2+\sigma_{x\x{m}}^2)^2-\rho_\x{m}^2\sigma_{x\x{m}}^2(\sigma^2+\sigma_{x\x{m}}^2)+\rho_\x{m}^2\sigma_{x\x{m}}^2\sigma_{y\x{f}}^2}{(\sigma^2+\sigma_{x\x{m}}^2)^2}\right)+\frac{1}{2}\sigma_{y\x{f}}^2\frac{\rho_\x{m}\sigma_{x\x{m}}\sigma_{y\x{m}}}{\sigma^2+\sigma_{x\x{m}}^2}+\frac{1}{4}\sigma_{y\x{f}}^2
\\3. \ & \sigma_{y\x{f}}^2(t+1) = \sigma_{x\x{m}}^2\left(\frac{\sigma^2(\sigma^2+\sigma_{x\x{m}}^2)+\sigma_{ x\x{m}}^2\sigma_{y\x{f}}^2}{(\sigma^2+\sigma_{x\x{m}}^2)^2}\right).
\end{align*}
When females acquire their song by learning from their mothers, the distribution of songs does not change over time because all females have equal mating success, so the variance of preferences stays the same. Note that this could happen if females observe the mating choices of their mothers and copy those choices or if preference is a sex-linked gene that only females have. 

Given a distribution of traits among adults, how the variance of songs changes does not depend on how preference is acquired. Likewise, how the variance of preferences changes does not depend on how song is acquired. However, in order to keep track of how these quantities change over many generations, we also need to keep track of how $\rho_\x{m}$ and $\rho_\x{f}$, or equivalently $C_\x{m}$ and $C_\x{f}$, change over time. These dynamics depend both on how songs are acquired and on how preferences are acquired. Note that there can only be a correlation between the traits if one or other is genetic, since otherwise each gender only has one of the two traits.

There are nine possible combinations of the three ways each trait can be acquired, which we will number from $1$ to $9$: 
\newline
\begin{equation*}
\begin{tabular}{|r|c|c|c|}
\hline & song learned obliquely  & song genetic & song from father
\\\hline pref from mother & 1 & 2 & 3
\\\hline pref genetic & 4 & 5 & 6
\\\hline pref from father & 7 & 8 & 9
\\\hline
\end{tabular}
\end{equation*}
\newline

In the Supporting Information, we derive the full dynamics of $\sigma_x^2$, $\sigma_y^2$, and $C$ for each of these modes of acquisition. Here we will show as an example mode $2$, the case where song is genetic and a female acquires the preference of its mother. Since both genders acquire song in the same way, we can drop the gender-specific subscripts on $\sigma_x^2$, and since only females have preferences, we can drop the gender-specific subscripts on $\sigma_y^2$ without losing track of which gender we are referring to. In this case, since males only have songs, $C_\x{m}=0$. However, females have both traits and, using Equation \ref{covariance}, the covariance between those traits among females of the next generation will be 
\begin{align*}
C_\x{f}(t+1)&=\frac{1}{2}\sigma_{x}^2\frac{\sigma_{y}^2}{\sigma^2+\sigma_{x}^2}+\frac{1}{2}\rho_\x{f}\sigma_{x}\sigma_{y}.
\end{align*}
Since 
Using the fact that $C_\x{f}=\rho_\x{f}\sigma_{x}\sigma_{y}$, we now have three equations for how $\sigma_{x}^2$, $\sigma_{y}^2$, and $C_\x{f}$ change over time:
\begin{align*}
\sigma_x^2(t+1)&=\frac{\sigma_x^2}{4}\left(\frac{\sigma^2(\sigma^2+\sigma_{x}^2)+\sigma_{x}^2\sigma_{y}^2}{(\sigma^2+\sigma_{x}^2)^2}+\frac{2C_\x{f}}{\sigma^2+\sigma_x^2}+1\right)
\\ \sigma_y^2(t+1)&=\sigma_y^2
\\ C_\x{f}(t+1)&=\frac{1}{2}\sigma_{x}^2\frac{\sigma_{y}^2}{\sigma^2+\sigma_{x}^2}+\frac{1}{2}C_\x{f}.
\end{align*}
%Consider the case where song is genetic and preference is learned from a female's father. In this case, males only have songs, so $C_\x{m}=0$. However, females have both traits and, using Equation \ref{covariance}, the covariance between those traits among females of the next generation will be 
%\begin{align*}
%C_\x{f}(t+1)&=\frac{1}{2}\sigma_{x\x{m}}^2\left(\frac{\sigma^2(\sigma^2+\sigma_{x\x{m}}^2)+\sigma_{x\x{m}}^2\sigma_{y\x{f}}^2}{(\sigma^2+\sigma_{x\x{m}}^2)^2}\right)+\frac{1}{2}\sigma_{x\x{m}}^2\frac{\rho_\x{f}\sigma_{x\x{f}}\sigma_{y\x{f}}}{\sigma^2+\sigma_{x\x{m}}^2}.
%\end{align*}
%Using the fact that $C_\x{f}=\rho_\x{f}\sigma_{x\x{f}}\sigma_{y\x{f}}$, we now have three equations for how $\sigma_{x}^2$, $\sigma_{y}^2$, and $C_\x{f}$ change over time:
%\begin{align*}
%\sigma_{x}^2(t+1)&=\frac{\sigma_x^2}{4}\left(\frac{\sigma^2(\sigma^2+\sigma_{x}^2)+\sigma_{x}^2\sigma_{y}^2}{(\sigma^2+\sigma_{x}^2)^2}+\frac{2C_\x{f}}{\sigma^2+\sigma_x^2}+1\right)
%\\\sigma_{y}^2(t+1)& =\sigma_{x}^2\left(\frac{\sigma^2(\sigma^2+\sigma_{x}^2)+\sigma_{x}^2\sigma_{y}^2}{(\sigma^2+\sigma_{x}^2)^2}\right)
%\\C_\x{f}(t+1)&=\frac{1}{2}\sigma_{x}^2\left(\frac{\sigma^2(\sigma^2+\sigma_{x}^2)+\sigma_{x}^2\sigma_{y}^2}{(\sigma^2+\sigma_{x}^2)^2}\right)+\frac{1}{2}\sigma_{x}^2\frac{C_\x{f}}{\sigma^2+\sigma_{x}^2}.
%\end{align*}
These three recursion equations fully describe how $\sigma_x^2$, $\sigma_y^2$, and $C_\x{f}$ change over time. In the Supporting Information, we show that there are two stable equilibria of these dynamics: one stable equilibrium occurs at $\sigma_x^{2\star}=C_\x{f}^\star=0$ and another occurs when
\begin{align*}
\sigma_x^{2\star}&=\frac{3\sigma_y^2-5\sigma^2+\sqrt{9(\sigma_y^2)^2-30\sigma^2\sigma_y^2+(\sigma^2)^2}}{6}
\\ C_\x{f}^\star&=\frac{\sigma_x^{2\star}\sigma_y^2}{\sigma^2+\sigma_x^{2\star}}.
\end{align*}
In the Supporting Information, we derive the recursion equations describing how $\sigma_x^2$, $\sigma_y^2$, and $C$ change over time and find the stable equilibria of these equations for each of the nine possible modes of acquisition.

\section*{Results}
\subsection*{Variance of song distribution at equilibrium. }
The stable equilibria of all nine modes of acquisition are presented in Table \ref{equilibrium}. We will first focus on the equilibrium values of $\sigma_x^2$. As noted above, when song is learned from a randomly chosen adult male the variance of the distribution of songs does not change so its equilibrium value $\sigma_x^{2\star}$ is equal to its initial value $\sigma_x^2(0)$. Of the other six modes, there are three in which $\sigma_x^2$ can approach a non-zero equilibrium and three in which $\sigma_x^2$ always approaches $0$:
\begin{enumerate}
\item
In mode $2$ (song is genetic and preference is maternally learned), the system of recursion equations for $\sigma_x^2$ and $C$ is bistable ($\sigma_y^2$ does not change). There are two stable equilibria: one where $\sigma_x^{2\star}=C=0$ and another where $\sigma_x^{2\star}>0$, as long as $\sigma_y^2\geq\frac{5+2\sqrt{6}}{3}\sigma^2$. Otherwise $\sigma_x^{2\star}=C=0$ is the only stable equilibrium. If the initial values of $\sigma_x^2$ and $C$ are small, the dynamics will reach the first equilibrium, and if they are high enough, the dynamics will reach the second equilibrium. See Figure \ref{mode9} for an example of trajectories that lead to each of these equilibria. 
\item In mode $3$ (song is paternally learned and preference is maternally learned), the recursion equations of $\sigma_x^2$ and $C$ only have one stable equilibrium, $\sigma_x^{2\star}=\min\{\sigma_y^2-\sigma^2,0\}$, which is greater than $0$ as long as $\sigma_y^2>\sigma^2$.
\item In mode $5$ (song is genetic and preference is genetic), the system of recursion equations for $\sigma_x^2$, $\sigma_y^2$, and $C$ is bistable. There is a stable equilibria where $\sigma_x^{2\star}=\sigma_y^{2\star}=C=0$, which is reached is the initial values of $\sigma_x^2$, $\sigma_y^2$, and $C$ are small. If the initial values are high enough, all three will increase indefinitely. (See Figure \ref{mode7}.)
\item In modes $6$, $8$, and $9$ the only stable equilibrium occurs when $\sigma_x^{2\star}=0$. (See Figures \ref{mode8}, \ref{mode1}), and \ref{mode2}). 
\end{enumerate}

For both traits, there are the most modes in which the variance of songs at equilibrium is greater than $0$ when the trait is maternally or obliquely learned, the second most when the trait is genetic, and the least when the trait is paternally learned. The rows and columns of Table \ref{equilibrium} are ordered so that $\sigma_x^{2\star}=0$ in cells that are lower and to the right. When song is obliquely learned, $\sigma_x^{2\star}=\sigma_x^2(0)>0$, regardless of how preference is acquired. When song is genetic, two of three ways that preference is acquired can give rise to $\sigma_x^{2\star}>0$. When song is paternally learned, only one way that preference is acquired can give rise to $\sigma_x^{2\star}>0$. Similarly, when preference is maternally learned, all three ways that song is acquired can give rise to $\sigma_x^{2\star}>0$. When preference is genetic, two of three ways that song is acquired can give rise to $\sigma_x^{2\star}>0$. When preference is paternally learned, only one way that song is acquired can give rise to $\sigma_x^{2\star}>0$.  

Conversely, the actual amount of variance in songs is greater when song is paternally learned than when song is genetic. We again split the $6$ modes in which song is not obliquely learned according to whether $\sigma_x^{2\star}>0$ or $\sigma_x^{2\star}=0$. Of the two modes in which preference is maternally learned, there are some choices of $\sigma^2$ that lead to $\sigma_x^{2\star}>0$ and some that lead to $\sigma_x^{2\star}=0$. Of the former, $\sigma_x^{2\star}$ is greater when song is paternally learned than when song is genetic. Of the latter, $\sigma_x^2(10)$ is greater when song is paternally learned than when song is genetic (compare the pink and gray lines in Figure \ref{sigmax2_sigma2}A and B). Of the modes in which preference is genetic, $\sigma_x^2(10)$ is greater and it takes a larger number of generations for $\sigma_x^2$ to drop below $5\times10^{-3}$ when song is paternally learned than when song is genetic (compare the green and red lines in Figure \ref{sigmax2_sigma2}B and C). Of the modes in which preference is paternally learned, $\sigma_x^2(10)$ is greater and it takes a larger number of generations for $\sigma_x^2$ to drop below $5\times10^{-3}$ when song is paternally learned than when song is genetic (compare the orange and blue lines in Figure \ref{sigmax2_sigma2}B and C).

%Similarly, when preference is learned from a female's mother, the equilibrium value $\sigma_y^{2\star}=\sigma_y^2(0)$. 

\subsection*{Female choosiness. }
To analyse the effect of female choosiness, we again split the modes of acquisition into two groups, depending on whether $\sigma_x^2$ reaches an equilibrium that is greater than $0$ or equal to $0$. For those modes in which $\sigma_x^{2\star}>0$ ($2$ and $3$), $\sigma_x^{2\star}$ decreases as a function of $\sigma^2$ (Figure \ref{sigmax2_sigma2}). In other words, when females are more choosy ($\sigma^2$ is smaller), there is greater diversity in male songs at equilibrium ($\sigma_x^{2\star}$ is larger). For those modes where $\sigma_x^{2\star}=0$ and where song is genetic ($2$, $5$, and $8$), we find that the transient amount of variance $\sigma_x^2(10)$ also decreases as a function of $\sigma^2$ (Figure \ref{sigmax2_sigma2}). For those modes where $\sigma_x^{2\star}=0$ and where song is paternally learned, ($3$, $6$ or $9$), $\sigma_x^2(10)$ does not strictly decrease with $\sigma^2$, but it is highest at small values of $\sigma^2$ (Figure \ref{sigmax2_sigma2}).  (We show $\sigma_x^{2\star}$, $\sigma_x^2(10)$, $\sigma_y^{2\star}$, and $\sigma_y^2(10)$ as a function of $\sigma^2$, as well as of initial values $\sigma_x^2(0)$ and $\sigma_y^2(0)$, in Figures \ref{sigmax2_full} and \ref{sigmay2_full}.)
%Finally, we measure the number of generations it takes before $\sigma_x^2$ is less than a threshold of $5\times 10^{-3}$. For those cases where song is genetic ($2$, $5$ and $8$), this number decreases as a function of $\sigma^2$ (Figure \ref{sigmax2_sigma2}). On the other hand, for those cases where song is paternally learned ($6$ and $9$), this number tends to increase as a function of $\sigma^2$. 

\subsection*{Alternate equilibria. }
Until this point, we have assumed that both traits are initially normally distributed. With the addition of a Gaussian preference function, this ensures that both traits continue to be normally distributed. Here we explore how the dynamics given by the mating rules in Equation \ref{model} are affected when we use either trait that are not normally distributed or a non-Gaussian preference function. In this case, it is not possible to write down an analytical expression for the distribution of traits among the offspring generation. Instead, it becomes necessary to find the distribution of traits among offspring numerically. To do so, we restrict both traits to be within a range $[-M,M]$ and we establish a partition of this range with an interval size of $\delta$:
$$
S=\{-M,-M+\delta,\dots,-2\delta,-\delta,0,\delta,2\delta,\dots,M-\delta,M\}.$$
For example, if $M=10$ and $\delta=0.01$, we consider a partition $$S=\{-10,-9.99,\dots,-0.02,-0.01,0,0.01,0.02,\dots,9.99,10\}$$
We only consider trait values that are in this partition, for both songs and preferences. For adults of each gender, we then construct a probability distribution over trait values in this partition:
\begin{align*}
P_\x{m}(v) &\text{ such that } \sum_{x\in S}\sum_{y \in S}P(x,y)=1
\\ \text{ and } P_\x{f}(v) &\text{ such that } \sum_{x\in S}\sum_{y \in S}P_\x{f}(x,y)=1
\end{align*}
We also have to define a preference function $f_y$ over this partition, such that $\sum_{x\in S}f_y(x)=1$. Given initial trait distributions and a preference function, we use a slightly modified version of Equation \ref{model} to describe the probability of a particular pair mating. If $u=(x_\x{m},y_\x{m},x_\x{f},y_\x{f})^T$, the probability of a pair with traits $(x_\x{m},y_\x{m})$ and $(x_\x{f},y_\x{f})$ mating is 
\begin{equation}
P_\text{mate}(u)=\frac{P_\x{f}(v_\x{f})P_\x{m}(v_\x{m})f_{y_\x{f}}(x_\x{m})}{Z_{y_\x{f}}}, \label{model_numerical}
\end{equation}
where $Z_{y_\x{f}}=\sum_{x\in S}\sum_{y\in S}P_\x{m}((x,y))f_{y_\x{f}}(x_\x{m})$. We can find this distribution numerically and then sum over the appropriate dimensions of $P_\text{mate}(u)$ to find the distribution of each trait in the offspring generation, for each of the nine modes of acquisition.

We can use this scheme with the same normal trait distributions and Gaussian preference functions as we used above, with the slight modification that we have to normalize the distributions and preference function. For example, instead of using 
\begin{align*}
P_\x{m}(v)=\frac{1}{2\pi\sqrt{|\Sigma_\x{m}|}}\exp\left(-\frac{1}{2}(v-\mu_\x{m})^T\Sigma_\x{m}^{-1}(v-\mu_\x{m})\right)
\end{align*}
for any $v\in\R^2$, we use
\begin{align*}
P_\x{m}(v)=\frac{\frac{1}{2\pi\sqrt{|\Sigma_\x{m}|}}\exp\left(-\frac{1}{2}(v-\mu_\x{m})^T\Sigma_\x{m}^{-1}(v-\mu_\x{m})\right)}{\sum_x\sum_y\frac{1}{2\pi\sqrt{|\Sigma_\x{m}|}}\exp\left(-\frac{1}{2}(v-\mu_\x{m})^T\Sigma_\x{m}^{-1}(v-\mu_\x{m})\right)}
\end{align*}
for $v=\in S\times S$. We can normalize a normal distribution of preferences and a Gaussian preference function similarly. 

Our purpose in using this numerical process to see how the distribution of the traits change over time is to explore the robustness of the previous results to using different initial trait distributions and a different preference function. In particular, we will use step functions for each of these.  Step functions introduce very sharp boundaries between the traits that have probability $0$ of being found in the population and those that have probability greater than $0$ of being found in the population. These boundaries will persist over the generations, introducing boundary effects at the edges of the distribution. Therefore, we introduce a small probability that a male's song will either mutate or that he will introduce errors in his song as he learns. This introduces an additional step in the model. Let $P_\x{m}\left(x|t+\Delta t\right)$ be the distribution of songs within male offspring found by using Equation \ref{model_numerical} and a particular mode of acquisition. For example, if song is genetic, $P_\x{m}\left(x|t+\Delta t\right)$ would be the distribution of the genes for songs among male zygotes resulting from mating pairs. Before the male is born and produces his first song, however, mutations may occur that slightly alter this distribution. On the other hand, if song is learned, $P_\x{m}\left(x|t+\Delta t\right)$ would be the distribution of songs heard by young males. Young males may make small errors in how the hear or produce the songs they have just heard, so that the distribution of songs when the members of the new generation themselves reproduce may be slightly different. In either case, we assume there is a small ``mutation" rate, $\mu$, so that the probability of finding an adult from the new generation singing song $x$ will be 
\begin{align*}
P_\x{m}(x|t+1)&=(1-\mu)P_\x{m}(x|t+\Delta t)+\frac{\mu}{2}P_\x{m}(x-\delta|t+\Delta t)+\frac{\mu}{2}P_\x{m}(x+\delta|t+\Delta t).
\end{align*}  
As long as $M$ is large, $\delta$ is small, and $\mu$ is small, the numerical scheme outlined above provides an excellent approximation to the analytical distribution we derived above. (See Figure XXX.)

We will focus here on how to construct a step preference function, but the process is the same for constructing probability distributions of each trait that are step functions. Just as the variance of a Gaussian preference function, $\sigma^2$, was a critical parameter in the analyses above, it will be a critical parameter as we explore the dynamics given by Equation \ref{model_numerical}. For a given $\sigma^2$, to completely define a step function we have to specify the widths of the steps, which we will do by choosing two numbers $m>n>0$ such that $m,n\in S$. There will be three steps, covering the intervals $[-m,-n)$, $[-n,n]$, and $(n,m]$. The step function will be $0$ outside of these intervals. Specifically, we will define three subpartitions of $S$:
\begin{align*}
S_1 & = \{-m,-m+\delta,...,-n-2\delta,-n-\delta\}
\\ S_2&=\{-n,-n+\delta,-n+2\delta,\dots,-\delta,0,\delta,\dots,n-2\delta,n-\delta,n\}
\\ S_3&=\{n+\delta,n+2\delta,\dots,m-\delta,m\}
\end{align*}
We will then construct a step function $P(x)$ over $S$ such that $P(x)=p_1$ for $x\in S_1$ or $S_3$ and $P(x)=p_2$ for $x\in S_3$. To ensure that $\sum_{x\in S}P(x)=1$ and $\sum_{x\in S}x^2P(x)=\sigma^2$, we require that 
\begin{align*}
2|S_1|p_1+|S_2|p_2&=1
\\2\sum_{x\in S_1}x^2p_1+\sum_{x\in S_2}x^2p_2&=\sigma^2.
\end{align*}
Having fixed $S_1$ and $S_2$ this gives a system of two equations we can solve to find the appropriate values of $p_1$ and $p_2$. An illustration of this process is provided in Figure \ref{step_ex}. For a particular $\sigma^2$, we only use $m$ and $n$ that give $p_1$ and $p_2$ such that both are positive and $p_2>p_1$.

For these analyses we focus on $3$ and $7$: in mode $3$ preference is maternally learned and the variance in the distribution of songs reaches a nonzero equilibrium, and in mode $7$ song is learned obliquely and the variance in the distribution of preferences reaches a nonzero equilibrium. In each, the variance of the trait of interest (song in the former and preference in the latter) does not depend on the shape of the preference being used (Figure \ref{effect_of_step_function}). In other words, the results we showed above are robust to using different preference functions.

Using initial distributions of each trait also do not strongly affect the variance of the traits of interest in these modes (Figure \ref{effect_of_step_dist}). Using a step function as the distribution of songs does change the equilibrium variance of the preference distribution in mode $7$ a little. However, the trend in this variance as a function of the variance of the preference function is very similar. Otherwise, regardless of the widths of the steps used in the step function for either trait distribution, the equilibrium variance of the song distribution in mode $3$ and the equilibrium variance of the preference distribution in mode $7$ are almost exactly the same as if the initial conditions were normal. 

Up to now, we have been focusing on the variance of the trait distributions at equilibrium because they have been unimodally distributed and variance is a convenient was of measuring the shape of a unimodal distribution. However, when we use step functions for the initial conditions of either trait, the distribution of songs in mode $3$ does not approach a unimodal distribution. Rather, the equilibrium distribution exhibits multiple peaks (Figure \ref{effect_of_step_dist}). 

These peaks occur because of how the number of males a female finds attractive is distributed. Remember that $Z_{y_\x{f}}=\sum_{x\in S}\sum_{y\in S}P_\x{m}((x,y))f_{y_\x{f}}(x_\x{m})$ indicates the total number of males a female with preference $y$ finds attractive. When the distribution of male songs follows a step function, rather than a normal distribution, there are somewhat more males singing the ``average" song and and more singing extreme songs, but fewer males singing songs in between. Consequently, females that prefer average songs and extreme songs have many options, but females that prefer songs in between have fewer options. Their desperation means that males with ``in-between" songs have disproportionate mating success. In-between males increase in frequency until there are enough of them to sate the demand of previously desperate females. As this peak is established, there are troughs on either side that are then favored because the females that prefere those songs do not have many options. This process of peaks appearing continues until the distribution reaches an equilibrium, where the desperation of females is balanced XXXX

\section*{Discussion}

%There are two cases in which song is genetic that can maintain song diversity ($2$ and $5$), but only one in which song is paternally learned than can maintain song diversity ($3$). On the other hand, there is a higher amount of song diversity present in those cases where song is paternally learned compared to those cases where song is genetic (compare mode $3$ to $2$, $6$ to $5$, and $9$ to $8$). This leads to the following prediction: of those species that have multiple song types, more should have genetic song than paternally learned song, but there should be more song types in species in which song is paternally learned than in species in which song is genetic. 

%When we use initially normally distributed traits, the traits continue to be normally distributed until they reach equilibrium. When we use traits that are initially distributed according to a step function, in some cases we find distributions of songs at equilibrium that exhibit multiple peaks. Rather than interpreting the presence of multiple discrete song types as a transient condition that will eventually be replaced by one in which only one song is present, it is possible that such a pattern can be maintained even by the simple mating process we model here. While we found this pattern by using initial distributions that followed step functions, other initial conditions could lead to the same patterns. The important thing is not the exact shape of the initial conditions, since the step function shape disappears after only a few generations. 


\newpage

\bibliographystyle{plainnat}
\bibliography{song_learning_evolution}

\begin{table}
\caption{\label{variables} Table of variables used in the text. In this table, we only give the version of each variable with a male-specific subscript for brevity. For example, while we use both $\sigma_{x\x{m}}^2$ and $\sigma_{x\x{f}}^2$ in the text, we only include $\sigma_{x\x{m}}^2$ in the table.}
\begin{tabular}{lllll}
Variable & Interpretation
\\\hline $C_\x{m}=\rho_\x{m}\sigma_{x\x{m}}\sigma_{y\x{m}}$ & covariance of traits among adult males
\\ $C^\star$ & equilibrium covariance of traits
\\ $M$ & half-width of interval over which we find probability distribution numerically
\\ $\mu$ & mutation rate 
\\$\mu_\x{m}=(\mu_{x\x{m}},\mu_{y\x{m}})^T$ & vector of average values of each trait among adult males 
\\$m,n$ & give widths of step function
\\ $\rho_\x{m}=\frac{C_\x{m}}{\sigma_{x\x{m}}\sigma_{y\x{m}}}$ & correlation between traits among adult males
\\$\Sigma_\x{m}$ & covariance matrix of traits among adult males
\\$\Sigma_\text{mate}$ & covariance matrix of traits among mating adults
\\$\sigma_{x\x{m}}^2$ & variance of songs among adult males
\\$\sigma_{x}^{2\star}$ & equilibrium variance of songs
\\$\sigma_{y\x{m}}^2$ & variance of preferences among adult males
\\$\sigma_y^{2\star}$ & equilibrium variance of preferences
\\$\sigma^2$ & variance of preference function
\\$u=(x_\x{m},y_\x{m},x_\x{f},y_\x{f})^T$ & vector of both traits in both genders
\\$v=(x,y)^T$ & vector of traits
\\$x$ & song
\\$y$ & preference
\\$Z_y=\int_{\R^2}P_\x{m}(v_\x{m})f_{y}(x_\x{m})dv_\x{m}$ & integral of males a female with preference $y$ finds attractive
\end{tabular}
\end{table}


\begin{table}
\caption{\label{equilibrium}Here we show the stable equilibria of the nine possible modes of acquisition. Stable equilibria are indicated with a ${}^\star$ and initial values are indicated with $(0)$. For example $\sigma_x^{2\star}$ is a stable equilibrium of $\sigma_x^2$ and $\sigma_x^2(0)$ is the initial value of $\sigma_x^2$. There are two modes in which the system of recursion equations for $\sigma_x^2$ and $C$ are bistable: when song is genetic and preference is maternally learned and when both song and preference are genetic. Both stable equilibria are included in the table. The second equilibrium when song is genetic and preference is maternally learned only exists when $\sigma_y^2\geq\frac{5+2\sqrt{6}}{3}\sigma^2$. The rows and columns are ordered such that the equilibrium value $\sigma_x^{2\star}$ is smaller in cells that are lower and to the right. }
\begin{tabular}{|r|l|l|l|}
\hline & song learned obliquely  & song genetic & song from father
\\\hline pref from mother  & $\sigma_x^{2\star}=\sigma_x^2(0)$ & $\sigma_x^{2\star}=0, \ \frac{3\sigma_y^2-5\sigma^2+\sqrt{9(\sigma_y^2)^2-30\sigma^2\sigma_y^2+(\sigma^2)^2}}{6}$ & $\sigma_x^{2\star}=\min\{\sigma_y^2-\sigma^2,0\}$  
\\ 	& 	$\sigma_y^{2\star}=\sigma_y^2(0)$ 	& $\sigma_y^{2\star}=\sigma_y^2(0)$ 		  & $\sigma_y^{2\star}=\sigma_y^2(0)$   
\\ & $ C^\star=0$ &   $ C^\star=\frac{\sigma_x^{2\star}\sigma_y^{2\star}}{\sigma^2+\sigma_x^{2\star}}$  & $ C^\star=0$
\\\hline pref genetic &  $\sigma_x^{2\star}=\sigma_x^2(0)$  & $\sigma_x^{2\star}=0,\ \infty$  & $\sigma_x^{2\star}=0$                      
\\  		&  $\sigma_y^{2\star}=0$	& $\sigma_y^{2\star}= 0 , \ \infty$ 	  & $\sigma_y^{2\star}=0$  
\\ & $ C^\star=0$   & $ C^\star=0, \ \infty$        & $ C^\star=0$          
\\\hline pref from father & $\sigma_x^{2\star}=\sigma_x^2(0)$ & $\sigma_x^{2\star}=0$  & $\sigma_x^{2\star}=0$                       
\\  			& $\sigma_y^{2\star}=\frac{\sigma_x^2(\sigma^2+\sigma_x^2)}{2\sigma_x^2+\sigma^2}$	  & $\sigma_y^{2\star}=0$  & $\sigma_y^{2\star}=0$                       
\\ & $ C^\star=0$ & $ C^\star=0$ & $ C^\star=0$
\\\hline
\end{tabular}
\end{table}

\begin{figure}
\includegraphics[width=6.5in]{/Users/eleanorbrush/Desktop/sigmax2_sigma2.pdf}
\caption{\label{sigmax2_sigma2} The variance of the song trait at equilibrium, $\sigma_x^{2\star}$, decreases as a function of $\sigma^2$. The variance of the song trait after a few generations, $\sigma_x^2(10)$, decreases or is non-monotonic as a function of $\sigma^2$. There are two possibilites for the dynamics of $\sigma_x^2$: the equilibrium value of $\sigma_x^2$ can either be greater than $0$ or equal to $0$. In this figure we ignore the modes in which song is learned obliquely. In all three panels, the variance of the preference function, $\sigma^2$, is on the horizontal axis. In the first panel, we show $\sigma_x^{2\star}$ for two other modes in which $\sigma_x^{2\star}>0$ ($2$ and $3$). (In mode $5$, $\sigma_x^2$ approaches infinity, which we do not show here.) In the second row, we show $\sigma_x^2(10)$  for those modes in which $\sigma_x^{2\star}=0$ ($2$, $3$, $5$, $6$, $8$, and $9$). In the third row, we show the number of generations before $\sigma_x^2$ is less than $5\times10^{-3}$ for those modes in which $\sigma_x^{2\star}=0$. Note that for the modes $2$, $3$, and $5$, some parameters give $\sigma_x^{2\star}>0$ and others give $\sigma_x^{2\star}=0$, i.e. the grey and pink lines are split between the first and second panels and the green line only appears for some parameters in the second panel. Parameters: unless the parameter is being varied $\sigma_x^2(0)=0.8$, $\sigma_y^2(0)=4$, $\sigma^2=1$, $\rho(0)=0.6$. In this figure we use normally distributed traits and a Gaussian preference function. }
\end{figure}

\begin{figure}
\includegraphics[width=3.25in]{/Users/eleanorbrush/Desktop/step_function_example.pdf}
\caption{\label{step_ex}An example of how a step function is constructed with given variance. On the horizontal axis is the difference between a females preferred song $y$ and a male's song $x$. On the vertical axis is a female's preference. The red line shows a discretized version of a Gaussian preference function with variance $\sigma^2=1.5$. The blue line shows a step function with the same variance, with $m=4$ and $n=1$. Here we use a grid with a large $\delta$ for the purposes of illustrating how we find $p_1$ and $p_2$. However, in our analyses, we use a much smaller $\delta$. }
\end{figure}

\begin{figure}
\includegraphics[width=6.5in]{/Users/eleanorbrush/Desktop/effect_of_step_function.pdf}
\caption{\label{effect_of_step_function} Using a step preference function, instead of a Gaussian preference function, has barely any effect on the equilibrium distributions of song and preference. In the upper row, we show examples of preference functions that have the same variance, in the left panel $\sigma^2=0.3$ and in the right $\sigma^2=0.9$. The red line in each panel is a Gaussian with the variance given. Lines with the same color in the two panels have steps of the same width. There is no purple function in the left panel because no function with those widths can have the desired variance. In the lower left panel, we show how the equilibrium variance of the song distribution in mode $3$ (song is paternally learned and preference is maternally learned) as a function of $\sigma^2$. The colors of the lines correspond to the colors of the preference functions in the upper panels. The lines overlap, indicating that the shape of the preference function does not affect the variance of the equilibrium distribution of songs. In the lower right panel, we show how the equilibrium variance of the preference distribution in mode $7$ (song is learned obliquely and preference is paternally learned) as a function of $\sigma^2$. The colors of the lines correspond to the colors of the preference functions in the upper panels. There are differences between the lines, but they are overall very similar to each other, indicating that the shape of the preference function does not have a large effect on the variance of the equilibrium distribution of preferences. Parameters: $\mu=0.01$, $\rho(0)=0$, $\sigma_x^2(0)=0.8$, $\sigma_y^2(0)=2$, number of generations = $5000$. In this figure, both traits are initially normally distributed.}
\end{figure}

\begin{figure}
\includegraphics[width=6.5in]{/Users/eleanorbrush/Desktop/effect_of_step_distribution.pdf}
\caption{\label{effect_of_step_dist}The model allows for equilibrium distributions of songs that are multimodal. In this figure, we show the effect of using step functions as initial conditions for the two traits, while using a Gaussian preference function. In the upper row, we show initial distributions of song with variance $\sigma_x^2(0)=0.8$ and of preference with variance $\sigma_y^2(0)=2$. The red line in each panel is a Gaussian with the variance given. In the left column, we use the step functions indicated in the top left panel for the initial distributions of songs, while using a normal distribution of preferences. In the right column, we use the step functions indicated in the top right panel for the initial distributions of preferences, while using a normal distribution of songs.  In the second row, we show the equilibrium variance of the song distribution in mode $3$ (song is paternally learned and preference is maternally learned) as a function of $\sigma^2$. In the third row, we show the equilibrium variance of the preference distribution in mode $7$ (song is learned obliquely and preference is paternally learned) as a function of $\sigma^2$. In the bottom row, we show equilibrium distributions of song in mode $3$ for various initial conditions, where variance of the preference function is fixed at $\sigma^2=1.1$. Parameters: $\mu=0.01$, $\rho(0)=0$, $\sigma_x^2(0)=0.8$, $\sigma_y^2(0)=2$, $\sigma^2=1.1$ unless it is being varied, number of generations = $5000$. In this figure, the preference function was Gaussian.}
\end{figure}

\begin{figure}
\includegraphics[width=6.5in]{/Users/eleanorbrush/Desktop/peak_example.pdf}
\caption{\label{peak_example}  A multimodal distribution occurs when the distribution of $Z$, the total number of males available to females with preference $y$, diverges from a normal distribution. In A we show the initial distribution of songs in the population, black indicating a normal distribution and red a step function. In B we show the equilibrium distribution of songs following from those initial conditions. In C we show $Z$ given by the initial distributions of songs as a function of female preference. In D we show the difference in the two lines in C, the black line minus the red line, as a function of female preference. The peaks in this curve lead to peaks in the distribution of songs. Parameters: $\mu=0.01$, $\rho(0)=0$, $\sigma_x^2(0)=0.8$, $\sigma_y^2(0)=2$, $\sigma^2=0.9$, number of generations = $5000$. In this figure, we use a Gaussian preference function. 
}
\end{figure}

\clearpage{}
\renewcommand{\thesection}{}
\renewcommand{\thesection}{S}
\renewcommand{\thesubsection}{S\arabic{subsection}}
\renewcommand{\theequation}{S\arabic{equation}}
\renewcommand{\thetable}{S\arabic{table}}
\renewcommand{\thefigure}{S\arabic{figure}}
\setcounter{equation}{0}  
\setcounter{figure}{0}
\setcounter{table}{0}

\begin{figure}
\includegraphics[width=3.25in]{/Users/eleanorbrush/Desktop/mode9_dynamics}
\caption{\label{mode9}}
\end{figure}

\begin{figure}
\includegraphics[width=3.25in]{/Users/eleanorbrush/Desktop/mode7_dynamics}
\caption{\label{mode7}}
\end{figure}

\begin{figure}
\includegraphics[width=3.25in]{/Users/eleanorbrush/Desktop/mode8_dynamics}
\caption{\label{mode8}}
\end{figure}

\begin{figure}
\includegraphics[width=3.25in]{/Users/eleanorbrush/Desktop/mode1_dynamics}
\caption{\label{mode1}}
\end{figure}


\begin{figure}
\includegraphics[width=3.25in]{/Users/eleanorbrush/Desktop/mode2_dynamics}
\caption{\label{mode2}}
\end{figure}


\begin{figure}
\includegraphics[width=6.5in]{/Users/eleanorbrush/Desktop/sigmax2_full.pdf}
\caption{\label{sigmax2_full} The variance of the song trait $\sigma_x^{2}$ increases as a function of $\sigma_y^2(0)$ and decreases or is non-monotonic as a function of $\sigma^2$. There are two possibilites for the dynamics of $\sigma_x^2$: the equilibrium value of $\sigma_x^2$ can either be greater than $0$ or equal to $0$. In this figure we ignore the modes in which song is learned obliquely. In the top row, we show $\sigma_x^{2\star}$ for two other modes in which $\sigma_x^{2\star}>0$ ($2$ and $3$). (In mode $5$, $\sigma_x^2$ approaches infinity, which we do not show here.) In the second row, we show $\sigma_x^2(10)$  for those modes in which $\sigma_x^{2\star}=0$ ($2$, $5$, $6$, $8$, and $9$). In the third row, we show the number of generations before $\sigma_x^2$ is less than $5\times10^{-3}$ for those modes in which $\sigma_x^{2\star}=0$. In the first column we show these quantities as a function of initial value $\sigma_x^{2}(0)$. In the second column we show these quantities as a function of initial value $\sigma_y^2(0)$. In the third column we show these quantities as a function of $\sigma^2$. Note that for the bistable modes, $2$ and $5$, some parameters give $\sigma_x^{2\star}>0$ and others give $\sigma_x^{2\star}=0$, i.e. the grey lines is split between the top and middle rows and the green line only appears for some parameters in the middle row. Parameters: unless the parameter is being varied $\sigma_x^2(0)=0.8$, $\sigma_y^2(0)=4$, $\sigma^2=1$, $\rho(0)=0.6$. In this figure we use normally distributed traits and a Gaussian preference function. } 
\end{figure}

\begin{figure}
\includegraphics[width=6.5in]{/Users/eleanorbrush/Desktop/sigmay2_full.pdf}
\caption{\label{sigmay2_full} The variance of the preference trait $\sigma_y^{2}$ increases as a function of $\sigma_y^2(0)$ and decreases or is non-monotonic as a function of $\sigma^2$. There are two possibilites for the dynamics of $\sigma_y^2$: the equilibrium value of $\sigma_y^2$ can either be greater than $0$ or equal to $0$. In this figure we ignore the modes in which preference is maternally learned. In the top row, we show $\sigma_y^{2\star}$ for the other mode in which $\sigma_y^{2\star}>0$ ($7$). (In mode $5$, $\sigma_y^2$ approaches infinity, which we do not show here.) In the second row, we show $\sigma_y^2(10)$  for those modes in which $\sigma_y^{2\star}=0$ ($4$, $5$, $6$, $8$, and $9$). In the third row, we show the number of generations before $\sigma_y^2$ is less than $5\times10^{-3}$ for those modes in which $\sigma_x^{2\star}=0$. In the first column we show these quantities as a function of initial value $\sigma_x^{2}(0)$. In the second column we show these quantities as a function of initial value $\sigma_y^2(0)$. In the third column we show these quantities as a function of $\sigma^2$. Note that for the bistable mode, $5$, some parameters give $\sigma_y^{2\star}>0$ and others give $\sigma_y^{2\star}=0$, i.e. the green line only appears for some parameters in the middle row.  Parameters: unless the parameter is being varied $\sigma_x^2(0)=0.8$, $\sigma_y^2(0)=4$, $\sigma^2=1$, $\rho(0)=0.6$. In this figure we use normally distributed traits and a Gaussian preference function.}
\end{figure}



\end{document}
